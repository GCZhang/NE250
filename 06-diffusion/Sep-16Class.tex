\documentclass[12pt]{article}
\usepackage[top=1in, bottom=1in, left=1in, right=1in]{geometry}

\usepackage{setspace}
\onehalfspacing

\usepackage{amssymb}
%% The amsthm package provides extended theorem environments
\usepackage{amsthm}
\usepackage{epsfig}
\usepackage{times}
\renewcommand{\ttdefault}{cmtt}
\usepackage{amsmath}
\usepackage{graphicx} % for graphics files

% Draw figures yourself
\usepackage{tikz} 

% writing elements
\usepackage{mhchem}

% The float package HAS to load before hyperref
\usepackage{float} % for psuedocode formatting
\usepackage{xspace}

% from Denovo Methods Manual
\usepackage{mathrsfs}
\usepackage[mathcal]{euscript}
\usepackage{color}
\usepackage{array}

\usepackage[pdftex]{hyperref}
\usepackage[parfill]{parskip}

% math syntax
\newcommand{\nth}{n\ensuremath{^{\text{th}}} }
\newcommand{\ve}[1]{\ensuremath{\mathbf{#1}}}
\newcommand{\Macro}{\ensuremath{\Sigma}}
\newcommand{\rvec}{\ensuremath{\vec{r}}}
\newcommand{\omvec}{\ensuremath{\hat{\Omega}}}
\newcommand{\sigs}{\ensuremath{\Sigma_s(\rvec,E'\rightarrow E,\omvec'\rightarrow\omvec)}}
\newcommand{\el}{\ensuremath{\ell}}
\newcommand{\sigso}{\ensuremath{\Sigma_{s,0}}}
\newcommand{\sigsi}{\ensuremath{\Sigma_{s,1}}}
%---------------------------------------------------------------------------
%---------------------------------------------------------------------------
\begin{document}
\begin{center}
{\bf NE 250, F15 \\
September 16, 2015}
\end{center}

We have now eliminated the dependence of $\phi$ on $\omvec$, so let's take a look at the two equations at
which we arrived last time. In 1D,

\begin{equation*}
\frac{1}{v}\frac{\partial\phi(z,E,t)}{\partial t} = S(z,E,t) + 
\int^{\infty}_0dE'\Sigma_s(E'\rightarrow E)\phi(z,E,t) - 
\Sigma_t(E)\phi(z,E,t) - \frac{\partial}{\partial z}J(z,E,t)
\end{equation*}

\begin{equation*}
\frac{1}{v}\frac{\partial J(z,E,t)}{\partial t} = S_1(z,E,t) + 
\int^{\infty}_0dE'\bar{\mu_0}\Sigma_s(E'\rightarrow E)J(z,E,t) - 
\Sigma_t(E)J(z,E,t) - \frac{1}{3}\frac{\partial}{\partial z}\phi(z,E,t)
\end{equation*}

In 3D,

\begin{equation*}
\frac{1}{v}\frac{\partial\phi(\rvec,E,t)}{\partial t} = S(\rvec,E,t) + 
\int^{\infty}_0dE'\Sigma_s(E'\rightarrow E)\phi(\rvec,E,t) - 
\Sigma_t(E)\phi(\rvec,E,t) - \nabla\cdot\vec{J}(\rvec,E,t)
\end{equation*}

\begin{equation*}
\frac{1}{v}\frac{\partial \vec{J}(\rvec,E,t)}{\partial t} = S_1(\rvec,E,t) + 
\int^{\infty}_0dE'\bar{\mu_0}\Sigma_s(E'\rightarrow E)\vec{J}(\rvec,E,t) - 
\Sigma_t(E)\vec{J}(\rvec,E,t) - \frac{1}{3}\nabla\phi(\rvec,E,t)
\end{equation*}

Assuming that all neutrons have the same energy (``one-speed approximation"),

\begin{equation*}
\frac{1}{v}\frac{\partial\phi(\rvec,t)}{\partial t} = S(\rvec,t) + 
\Sigma_s\phi(\rvec,t) - 
\Sigma_t\phi(\rvec,t) - \nabla\cdot\vec{J}(\rvec,t)
\end{equation*}

\begin{equation*}
\frac{1}{v}\frac{\partial \vec{J}(\rvec,t)}{\partial t} = S_1(\rvec,t) + 
\bar{\mu_0}\Sigma_s\vec{J}(\rvec,t) - 
\Sigma_t\vec{J}(\rvec,t) - \frac{1}{3}\nabla\phi(\rvec,t)
\end{equation*}

Now, assume that the source is isotropic:

\begin{equation*}
S(\rvec,\omvec,t) = \frac{S(\rvec,t)}{4\pi}
\end{equation*}

\begin{equation*}
S_1(\rvec,\omvec,t) = \int_{4\pi}d\omvec S(\rvec,\omvec,t)\omvec 
= \frac{S(\rvec,t)}{4\pi} \int_{4\pi}d\omvec\omvec = 0
\end{equation*}

Rearranging the current equation gives:

\begin{equation*}
\frac{1}{|\vec{J}(\rvec,t)|}\frac{\partial\vec{J}(\rvec,t)}{\partial t} 
= \frac{\bar{\mu_0}\Sigma_sv\vec{J}(\rvec,t)}{|\vec{J}(\rvec,t)|} - v\Sigma_t 
- \frac{v}{3|\vec{J}(\rvec,t)|}\nabla\phi(\rvec,t)
\end{equation*}

In general, $\frac{1}{|\vec{J}(\rvec,t)|}\frac{\partial\vec{J}(\rvec,t)}{\partial t} \ll v\Sigma_t$, so we
assume/approximate $\frac{1}{|\vec{J}(\rvec,t)|}\frac{\partial\vec{J}(\rvec,t)}{\partial t}\approx 0$. The
collision frequency $v\Sigma_t$ is typically on the order of $10^5$ sec${^-1}$ or larger, so only an
extremely rapid time variation of the current would invalidate this assumption (such rapid changes are
very rarely encountered in reactor dynamics). The current equation then becomes

\begin{equation*}
\vec{J}(\rvec,t) = \frac{-1}{3(\Sigma_t - \bar{\mu_0}\Sigma_s)}\nabla\phi(\rvec,t),
\end{equation*}

which is known as Fick's law. Let us define the diffusion coefficient as

\begin{equation*}
D(\rvec) = \frac{1}{3(\Sigma_t - \bar{\mu_0}\Sigma_s)} = \frac{1}{3\Sigma_{tr}},
\end{equation*}

where $\Sigma_{tr} = \Sigma_t - \bar{\mu_0}\Sigma_s$ is the ``transport" cross section. Also note that

\begin{equation*}
\bar{\mu_0} \approx \frac{2}{3A} > 0,
\end{equation*}

which means that scattering is forward-biased in the lab frame. For large $A$, $\bar{\mu_0}\rightarrow0$,
meaning that scattering is isotropic in the lab frame for large target nuclei. Plugging Fick's law back 
into the flux equation gives

\begin{equation*}
\frac{1}{v}\frac{\partial\phi(\rvec,t)}{\partial t} = S(\rvec,t) - \Sigma_a\phi(\rvec,t) + 
\nabla\cdot[D(\rvec)\nabla\phi(\rvec,t)],
\end{equation*}

which is the one-speed diffusion equation. The diffusion equation is of interest for several reasons. The
$P_1$ approximation is not valid at interface boundaries, in highly absorbing media, or near sources. The
diffusion equation is valid a few mean paths away from a source or a boundary but is also not valid in
highly absorbing media.


Now, assume an isotropic flux $\phi(\rvec,\omvec,t)$. The net current, $\vec{J}(\rvec,\omvec,t)$, is 
then zero.


Next, consider $J_x(\rvec,\omvec,t) > 0, J_y(\rvec,\omvec,t) = J_z(\rvec,\omvec,t) = 0$.


With the $P_1$ approximation, we have

\begin{equation*}
\phi(\rvec,\omvec,t)\approx\frac{1}{4\pi}\phi(\rvec,t)+\frac{3}{4\pi}\vec{J}(\rvec,E,t)\cdot\omvec
\end{equation*}

In order for both sides of this equation to be positive, it is required that

\begin{equation*}
-\vec{J}(\rvec,t)\cdot\omvec < \frac{1}{3}\phi(\rvec,t).
\end{equation*}

For this to be valid, we require that

\begin{equation*}
J_x < \frac{1}{3}\phi(\rvec,t).
\end{equation*}

To solve the diffusion equation, we need initial and boundary conditions.


Initial condition: $\phi(\rvec,0) = \phi_0(\rvec) \forall\rvec$


Interface boundary conditions:

% first column
\begin{minipage}[t]{0.5\textwidth}
\underline{Transport} \\
$\phi_1(\rvec_s,\omvec,t) = \phi_2(\rvec_s,\omvec,t)$ \\
$\forall\rvec_s \in S, S \equiv \partial V, \forall \omvec, \forall t$
\end{minipage}
%second column
\begin{minipage}[t]{0.5\textwidth}
\underline{Diffusion} \\
$\phi_1(\rvec_s,t) = \phi_2(\rvec_s,t)$ \\
$\forall\rvec_s \in S, S \equiv \partial V, \forall t$ \\
$\vec{J_1}(\rvec_s,t) = \vec{J_2}(\rvec_s,t)$
\end{minipage}

Vacuum boundary conditions:

% first column
\begin{minipage}[t]{0.5\textwidth}
\underline{Transport} \\
$\phi(\rvec_s,\omvec,t) = 0$ \\
$\forall\rvec_s \in S, S \equiv \partial V, \forall \omvec: \omvec\cdot\hat{e}_s<0, \forall t$
\end{minipage}
%second column
\begin{minipage}[t]{0.5\textwidth}
\underline{Diffusion} \\
\vspace{-10 mm}
\begin{align*}
\vec{J_-}(\rvec_s) &= \int_{2\pi^-}d\omvec\omvec\cdot\hat{e}_s
\left[\frac{\phi(\rvec,t)}{4\pi} + \frac{3}{4\pi}\omvec\cdot\vec{J}(\rvec,t)\right] \\
&= \frac{1}{4}\phi(\rvec_s,t) + \frac{D(\rvec)}{2}\nabla\phi(\rvec_s,t) \\
&= 0
\end{align*}
\end{minipage}

\begin{equation*}
\vec{J}(\rvec,t) = \int_{4\pi}d\omvec\omvec\phi(\rvec,\omvec,t) = \text{current at $\rvec$}
\end{equation*}

\begin{equation*}
J = \int_Sds\vec{J}(\rvec,t)\cdot\hat{e}_s = \text{current through a surface $S$}
\end{equation*}

\begin{equation*}
J_{net} = \int_{4\pi}d\omvec\omvec\cdot\hat{e}_s\phi(\rvec,\omvec,t) 
= \text{net current at $\rvec$ with respect to $\hat{e}_s$}
\end{equation*}

Scalar currents:

\begin{equation*}
J_{\pm} = \int_{2\pi^{\pm}}d\omvec\omvec\cdot\hat{e}_s\phi(\rvec,\omvec,t)
\end{equation*}

\begin{equation*}
2\pi^+ \equiv 0 < \theta < \pi/2, 2\pi^- \equiv \pi/2 < \theta < \pi
\end{equation*}

In 1D,

\begin{equation*}
J_-(z_s,t) = \frac{1}{4}\phi(z_s,t) + \frac{D(z_s)}{2}\frac{d\phi(z_s,t)}{dz}\Bigr|_{z_s} = 0
\end{equation*}

\begin{equation*}
\frac{d\phi(z_s,t)}{dz}\Bigr|_{z_s} = \frac{-\phi(z_s,t)}{2D(z_s)}
\end{equation*}

In order to use the diffusion approximation, we must accept that the above solution at the vacuum is 
wrong, necessitating the introduction of an extrapolation distance:

\begin{equation*}
\tilde{z}_s = z_s + 2D(z_s)
\end{equation*}

This allows us to obtain the correct solution away from the boundary.


\end{document}
