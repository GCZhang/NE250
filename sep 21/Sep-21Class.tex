\documentclass[12pt]{article}
\usepackage[top=1in, bottom=1in, left=1in, right=1in]{geometry}

\usepackage{setspace}
\onehalfspacing

\usepackage{amssymb}
%% The amsthm package provides extended theorem environments
\usepackage{amsthm}
\usepackage{epsfig}
\usepackage{times}
\renewcommand{\ttdefault}{cmtt}
\usepackage{amsmath}
\usepackage{graphicx} % for graphics files

% Draw figures yourself
\usepackage{tikz} 

% writing elements
\usepackage{mhchem}

% The float package HAS to load before hyperref
\usepackage{float} % for psuedocode formatting
\usepackage{xspace}

% from Denovo Methods Manual
\usepackage{mathrsfs}
\usepackage[mathcal]{euscript}
\usepackage{color}
\usepackage{array}

\usepackage[pdftex]{hyperref}
\usepackage[parfill]{parskip}

% math syntax
\newcommand{\nth}{n\ensuremath{^{\text{th}}} }
\newcommand{\ve}[1]{\ensuremath{\mathbf{#1}}}
\newcommand{\Macro}{\ensuremath{\Sigma}}
\newcommand{\rvec}{\ensuremath{\vec{r}}}
\newcommand{\omvec}{\ensuremath{\hat{\Omega}}}
\newcommand{\sigs}{\ensuremath{\Sigma_s(\rvec,E'\rightarrow E,\omvec'\rightarrow\omvec)}}
\newcommand{\el}{\ensuremath{\ell}}
\newcommand{\sigso}{\ensuremath{\Sigma_{s,0}}}
\newcommand{\sigsi}{\ensuremath{\Sigma_{s,1}}}
%---------------------------------------------------------------------------
%---------------------------------------------------------------------------
\begin{document}
\begin{center}
{\bf NE 250, F15 \\
September 21, 2015}
\end{center}

Please note: I posted a fairly different derivation of the TE and DE on the class page (called 15-19-te-de). This is how I taught it in NE 155 and I think it's both clearer and more thorough. This could be useful as you sort through things. 

Last class we started with the two equations we had arrived at using the $P_1$ approximation: one for scalar flux and one for current. We then made a few simplifications / assumptions:
\begin{itemize}
\item The angular flux can be represented by \textbf{linearly anisotropic} angular dependence (that is the $P_1$ expansion). 
The approximation is valid away from boundaries, away from neutron sources and sinks, and in media that are not highly absorbing.
\item one speed.
\item isotropic source.
\item neutron current density changes slowly on a time scale compared to the mean collision time:
\begin{align*}
\frac{1}{|\vec{J}(\rvec,t)|}&\frac{\partial\vec{J}(\rvec,t)}{\partial t} \ll v\Sigma_t \\ \frac{1}{|\vec{J}(\rvec,t)|}&\frac{\partial\vec{J}(\rvec,t)}{\partial t}\approx 0
\end{align*}
The collision frequency $v\Sigma_t$ is typically on the order of $10^5$ sec$^{-1}$ or larger, so only an extremely rapid time variation of the current would invalidate this assumption (such rapid changes are very rarely encountered in reactor dynamics).
\end{itemize}

All of that got us to \textit{Fick's law} and \textit{the one-speed diffusion equation}.  
%
\begin{equation*}
\frac{1}{v}\frac{\partial\phi(\rvec,t)}{\partial t} = S(\rvec,t) - \Sigma_a\phi(\rvec,t) + 
\nabla\cdot[D(\rvec)\nabla\phi(\rvec,t)],
\end{equation*}
%
We talked about the boundary conditions necessary to solve this equation last time. 

--------------------------------------------------------------\\
Now, let's revisit the the energy-dependent $P_1$ equations:

\begin{equation*}
\frac{1}{v}\frac{\partial\phi(\rvec,E,t)}{\partial t} = S(\rvec,E,t) + 
\int^{\infty}_0dE'\:\Sigma_s(E'\rightarrow E)\phi(\rvec,E,t) - 
\Sigma_t(E)\phi(\rvec,E,t) - \nabla\cdot\vec{J}(\rvec,E,t)
\end{equation*}

\begin{equation*}
\frac{1}{v}\frac{\partial \vec{J}(\rvec,E,t)}{\partial t} = S_1(\rvec,E,t) + 
\int^{\infty}_0dE'\:\bar{\mu_0}\Sigma_s(E'\rightarrow E)\vec{J}(\rvec,E,t) - 
\Sigma_t(E)\vec{J}(\rvec,E,t) - \frac{1}{3}\nabla\phi(\rvec,E,t)
\end{equation*}

From the $\frac{1}{|\vec{J}(\rvec,t)|}\frac{\partial\vec{J}(\rvec,t)}{\partial t}\ll v\Sigma_t$ 
assumption, $\frac{1}{v}\frac{\partial\vec{J}(\rvec,t)}{\partial t} = 0$.

From the isotropic source assumption, $S_1(\rvec,E,t) = 0$.

The assumption of isotripic scattering would give $\Sigma_{s1}(E' \rightarrow E)$, which results in
\[\vec{J}(\vec{r}, E, t) \approx - \frac{1}{3 \Sigma_t(\vec{r}, E)}\nabla \phi(\vec{r}, E, t)\:. \]
However, the assumption of isotropic scattering is often too strong for most reactor calculations.

Instead, we can simply define an energy-dependent diffusion coefficient as
\[D(\vec{r}, E) = \frac{1}{3} \biggl[ \Sigma_t(\vec{r}, E) - \frac{\int_0^{\infty} dE' \: \Sigma_{s1}(E' \rightarrow E)J_i(\vec{r}, E', t)}{J_i(\vec{r}, E', t)}\biggr]^{-1}\:,\]
which would automatically yield
\[\vec{J}(\vec{r}, E, t) = -D(\vec{r},E)\nabla \phi(\vec{r}, E, t)\:.\]
[Note that $i = x,y,z$. Technically we need to do each vector component separately and then recombine into $\vec{J}$. This is often skipped, but matches the formalism on pages 126-127 and 139 in Duderstadt.]\\
This is artificial because $D$ actually depends on $\vec{J}$.

One common way to avoid this is to neglect the anisotropic contribution to energy transfer in the scattering collision by saying
\[\Sigma_{s1}(E' \rightarrow E) = \Sigma_{s1}(E) \delta(E' - E)\]
such that
\[\int_0^{\infty} dE' \: \Sigma_{s1}(E' \rightarrow E)J_i(\vec{r}, E', t) = \bar{\mu_0} \Sigma_s(E) J_i(\vec{r}, E, t)\:.\]
And all of that gives
\[D(\vec{r}, E) = \frac{1}{3} \bigl[\Sigma_t(\vec{r}, E) - \bar{\mu_0} \Sigma_s(\vec{r}, E) \bigr]^{-1}\:.\]

Plugging this into the first $P_1$ equation gives

\begin{equation*}
\frac{1}{v}\frac{\partial\phi(\rvec,E,t)}{\partial t} = S(\rvec,E,t) + 
\int^{\infty}_0dE'\:\Sigma_s(E'\rightarrow E)\phi(\rvec,E',t) - 
\Sigma_t(E)\phi(\rvec,E,t) + \nabla\cdot[D(\rvec,E)\nabla\phi(\rvec,E,t)]
\end{equation*}


--------------------------------------------------------------\\
Now, let's make the \emph{one-group} (not one-speed) assumption. This means that we will integrate the
entire equation over all energy space $[\int_0^{\infty}dE(\cdot)]$.


``Group constants" are defined as follows:

\begin{equation*}
\Sigma_{t,1}(\rvec)=\frac{\int_0^{\infty}dE\Sigma_t(\rvec,E)\phi(\rvec,E,t)}{\int_0^{\infty}dE\phi(\rvec,E,t)}
= \text{effective cross section}
\end{equation*}

\begin{equation*}
\phi_1(\rvec,t) = \int_0^{\infty}dE\phi(\rvec,E,t) = \text{group flux}
\end{equation*}

Thus, $\int_0^{\infty}dE\Sigma_t(\rvec,E)\phi(\rvec,E,t) = \Sigma_{t,1}(\rvec)\phi_1(\rvec,t)$.

\begin{equation*}
\int_0^{\infty}dE\frac{1}{v}\frac{\partial\phi(\rvec,E,t)}{\partial t} = 
\frac{1}{v_1}\frac{\partial \phi_1(\rvec,t)}{\partial t}, 
\text{ where } \frac{1}{v_1} = \frac{\int_0^{\infty}dE\frac{1}{v}\phi(\rvec,E,t)}{\phi_1(\rvec,t)}
\end{equation*}

\begin{equation*}
\int_0^{\infty}dES(\rvec,E,t) = S_1(\rvec,t)
\end{equation*}

\begin{equation*}
\int_0^{\infty}dE\int_0^{\infty}dE'\Sigma_s(E'\rightarrow E)\phi(\rvec,E',t) =
\int_0^{\infty}dE'\Sigma_s(E')\phi(\rvec,E',t) = \sigsi\phi_1
\end{equation*}

\begin{equation*}
\int_0^{\infty}dE\nabla\cdot[D(\rvec,E)\nabla\phi(\rvec,E,t)]=\nabla\cdot[D_1(\rvec)\nabla\phi_1(\rvec,t)]
\end{equation*}

Note that although the subscripts used here for the one-group approximation are the same as those used in
the $P_1$ approximation derivation, the quantities are not the same. Combining all of the above terms, we
have

\begin{equation*}
\frac{1}{v_1}\frac{\partial \phi_1(\rvec,t)}{\partial t} = S_1(\rvec,t) - 
\Sigma_{a,1}(\rvec)\phi_1(\rvec,t) + \nabla\cdot[D_1(\rvec)\nabla\phi_1(\rvec,t)]
\end{equation*}

This is the \emph{one-group} diffusion equation.


We were talking about currents and boundary conditions at the end of class, so let's quickly revisit that. 

Scalar currents:

\begin{equation*}
J_{\pm} = \int_{2\pi^{\pm}}d\omvec\omvec\cdot\hat{e}_s\phi(\rvec,E,\omvec,t)
\end{equation*}

\begin{equation*}
2\pi^+ \equiv 0 < \theta < \pi/2, 2\pi^- \equiv \pi/2 < \theta < \pi
\end{equation*}

In 1D,

\begin{equation*}
J_-(z_s,t) = \frac{1}{4}\phi(z_s,t) + \frac{D(z_s)}{2}\frac{d\phi(z_s,t)}{dz}\Bigr|_{z_s} = 0
\end{equation*}

\begin{equation*}
\frac{d\phi(z_s,t)}{dz}\Bigr|_{z_s} = \frac{-\phi(z_s,t)}{2D(z_s)}
\end{equation*}

In order to use the diffusion approximation, we must accept that the above solution at the vacuum is 
wrong, necessitating the introduction of an extrapolation distance:

\begin{equation*}
\tilde{z}_s = z_s + 2D(z_s)
\end{equation*}

This allows us to obtain the correct solution away from the boundary.


\end{document}
