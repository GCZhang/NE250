\documentclass[12pt]{article}
\usepackage[top=1in, bottom=1in, left=1in, right=1in]{geometry}

\usepackage{setspace}
\onehalfspacing

\usepackage{amssymb}
%% The amsthm package provides extended theorem environments
\usepackage{amsthm}
\usepackage{epsfig}
\usepackage{times}
\renewcommand{\ttdefault}{cmtt}
\usepackage{amsmath}
\usepackage{graphicx} % for graphics files

% Draw figures yourself
\usepackage{tikz} 

% writing elements
\usepackage{mhchem}

% The float package HAS to load before hyperref
\usepackage{float} % for psuedocode formatting
\usepackage{xspace}

% from Denovo Methods Manual
\usepackage{mathrsfs}
\usepackage[mathcal]{euscript}
\usepackage{color}
\usepackage{array}

\usepackage[pdftex]{hyperref}
\usepackage[parfill]{parskip}

% math syntax
\newcommand{\nth}{n\ensuremath{^{\text{th}}} }
\newcommand{\ve}[1]{\ensuremath{\mathbf{#1}}}
\newcommand{\Macro}{\ensuremath{\Sigma}}
\newcommand{\rvec}{\ensuremath{\vec{r}}}
\newcommand{\vecr}{\ensuremath{\vec{r}}}
\newcommand{\omvec}{\ensuremath{\hat{\Omega}}}
\newcommand{\vOmega}{\ensuremath{\hat{\Omega}}}
\newcommand{\sigs}{\ensuremath{\Sigma_s(\rvec,E'\rightarrow E,\omvec'\rightarrow\omvec)}}
\newcommand{\el}{\ensuremath{\ell}}
\newcommand{\sigso}{\ensuremath{\Sigma_{s,0}}}
\newcommand{\sigsi}{\ensuremath{\Sigma_{s,1}}}
%---------------------------------------------------------------------------
%---------------------------------------------------------------------------
\begin{document}
\begin{center}
{\bf NE 250, F15\\
October 7, 2015 
}
\end{center}

Last time we derived the integral form of the TE:
\[
\psi(\rvec, \vOmega, E) =\int_0^{\infty} d\rho' \:\exp[-\int_0^{\rho'} d\rho'' \: \Sigma_t(\rvec-\rho''\vOmega,E)]q(\rvec-\rho'\vOmega,\vOmega,E)\]
%
We can reframe with with operators to think about how the levels of collisions build up the flux at a given point in phase space.\\
If we define $Q'$ as the integrated fixed source and\\
$K$ as the appropriate integral operator \\
then $K\psi$ is the neutron production by scattering and fission. \\
We can then see
\[\psi = K \psi + Q'\]
We can think about this equation 
\begin{align*}
\psi_0 &= Q' \quad \text{uncollided flux}\\
\psi_1 &= K \psi_0 \quad \text{flux of neutrons that have had one collision}\\
&\vdots \\
\psi_n &= K \psi_{n-1} \quad \text{flux of neutrons that have had n collisions} \\
\psi &= \sum_{j=0}^{\infty} \psi_j \quad \text{total flux distribution}
\end{align*}

---------------------------\\
If we have \textbf{isotropic scattering}, we can remove the angle dependence and up with a volume integral.
%
\begin{itemize}
\item The fission process is isotropic:
\[\frac{\chi(E)}{4\pi} \int_0^{\infty} dE' \: \nu(E') \Sigma_f(\rvec, E') \underbrace{\int_{4\pi} d\vOmega' \: \psi(\rvec, \vOmega', E')}_{\phi(\rvec,E')}\]
\item The fixed source becomes
\[S(\rvec, \vOmega, E) = \frac{S(\rvec, E)}{4 \pi}\]
\item and the scattering source becomes
\[\int_0^{\infty} dE' \int_{4\pi} d\vOmega' \: \frac{\Sigma_s(\rvec,E' \rightarrow E)}{4\pi} = \frac{1}{4\pi}\int_0^{\infty} dE'\:\Sigma_s(\rvec,E' \rightarrow E) \phi(\rvec,E')\]
\item The total source is the sum of these three, and is now independent of angle.
\end{itemize}






\end{document}
