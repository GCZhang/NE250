\documentclass[12pt]{article}
\usepackage[top=1in, bottom=1in, left=1in, right=1in]{geometry}

\usepackage{setspace}
\onehalfspacing

\usepackage{amssymb}
%% The amsthm package provides extended theorem environments
\usepackage{amsthm}
\usepackage{epsfig}
\usepackage{times}
\renewcommand{\ttdefault}{cmtt}
\usepackage{amsmath}
\usepackage{graphicx} % for graphics files
\usepackage{tabu}

% Draw figures yourself
\usepackage{tikz} 

% writing elements
\usepackage{mhchem}

% The float package HAS to load before hyperref
\usepackage{float} % for psuedocode formatting
\usepackage{xspace}

% from Denovo Methods Manual
\usepackage{mathrsfs}
\usepackage[mathcal]{euscript}
\usepackage{color}
\usepackage{array}

\usepackage[pdftex]{hyperref}
\usepackage[parfill]{parskip}

% math syntax
\newcommand{\nth}{n\ensuremath{^{\text{th}}} }
\newcommand{\ve}[1]{\ensuremath{\mathbf{#1}}}
\newcommand{\Macro}{\ensuremath{\Sigma}}
\newcommand{\rvec}{\ensuremath{\vec{r}}}
\newcommand{\vecr}{\ensuremath{\vec{r}}}
\newcommand{\omvec}{\ensuremath{\hat{\Omega}}}
\newcommand{\vOmega}{\ensuremath{\hat{\Omega}}}
\newcommand{\sigs}{\ensuremath{\Sigma_s(\rvec,E'\rightarrow E,\omvec'\rightarrow\omvec)}}
\newcommand{\el}{\ensuremath{\ell}}
\newcommand{\sigso}{\ensuremath{\Sigma_{s,0}}}
\newcommand{\sigsi}{\ensuremath{\Sigma_{s,1}}}
%---------------------------------------------------------------------------
%---------------------------------------------------------------------------
\begin{document}
\begin{center}
{\bf NE 250, F15\\
November 9, 2015 
}
\end{center}

Last time started talking about \textbf{Discretization of Angle} for TE. We looked at the 1-D, slab version of the TE within each group.\\
We mentioned that we need a collection of $\mu_n, w_n$, known as the angular quadrature set. \\
We frequently select $N$ is even and $\mu_n$ are symmetric about $\mu=0$.
%
\begin{align*}
\mu_{N+1-n} &= -\mu_n\:, \quad n = 1, 2, \dots, \frac{N}{2} \quad \mu_n > 0\\
w_n &= w_{N+1-n}
\end{align*}
%
If $N$ weren't even we would get $\mu_{(N+1)/2} = 0$, which makes the derivative term disappear. It can also make our BCs ambiguous.\\
With these conditions we get advantages in the boundary conditions as well.
%
\begin{align*}
\text{reflecting at } x=0 \quad &\psi_n(0) = \psi_{N+1-n}(0)\:, \quad n = 1, 2, \dots, \frac{N}{2}\\
\text{vacuum on right at } x=a \quad &\psi_n(a) = 0\:, \quad n = \frac{N}{2}+1, \frac{N}{2}+2, \dots, N
\end{align*}
%
We have a fair bit of freedom selecting an ordinate set. 
This DO formulation
\begin{itemize}
\item gives us $N$ ODEs and $N$ BCs
\item treats the BCs exactly as long as the direction is in the quadrature set
\item the scattering term is an approximation. You have to interpolate if you need $\psi$ along a non-included angle.
\end{itemize}

In slab, Legendre $P_N$ and Double Legendre $DP_N$ are the most popular (the most well known in 3D is the level symmetric, which is frequently referred to as $S_N$ just to be confusing). \\
We started the idea of $P_N$ in our derivation of the diffusion equation (which is the $P_1$ expansion of the TE). \\ 
The basic idea is that we can expand the angular flux in the same manner we expanded the scattering kernels before. \\
Note, however, that this is a different and distinct process. Now, we will use
\[
\psi(x, \mu) \approx \sum_{l=0}^{\infty} (2l+1)\phi_l(x)P_l(\mu)
\]
for the flux solution in total. This next part comes from Appendix D of L\&M. We again start with the within group, 1D TE
\[
\mu \frac{\partial \psi}{\partial x} + \Sigma_t(x)\psi(x,\mu) = \sum_{l=0}^{\infty} (2l+1) \Sigma_{s,l}(x) P_l(\mu)\phi_l(x) + S(x,\mu)
\]
where $S(x,\mu)$ is the group source given by
\[
S(x,\mu) = \sum_{l=0}^{\infty} (2l+1) P_l(\mu) \sum_{g' \neq g} \Sigma_{sl,gg'}(x) \phi_{lg'}(x) + \tilde{S}_g(x,\mu)
\]
with $\tilde{S}_g(x,\mu)$ being the external and/or fission source for the group. 
(I think I forgot to explicitly mention this last time; was also true for the within group in $S_N$).\\
As long as the series is infinite, we are not introducing any approximations.

We can use the Legendre expansion to make our TE an infinite set of coupled ODEs. We use the flux expansion in the equation, multiply by $\frac{1}{2}P_{l'}(\mu)$ and integrate over $-1 \leq \mu \leq 1$:
\begin{align*}
\sum_{l=0}^{\infty} &(2l+1)\biggl[ \frac{1}{2} \int_{-1}^1 d\mu\: \mu P_{l'}(\mu) P_l(\mu) \frac{d}{d x}\phi_l(x) + \frac{1}{2} \int_{-1}^1 d\mu\: P_{l'}(\mu) P_l(\mu) \Sigma_t(x)\phi_l(x) \biggr] \\
&= \sum_{l=0}^{\infty} (2l+1)\frac{\Sigma_{sl}}{2}\int_{-1}^1 d\mu\: P_{l'}(\mu) P_l(\mu) \phi_l(x) + s_{l'}(x)
\end{align*}
where
\[
s_{l'}(x) = \frac{1}{2} \int_{-1}^1 d\mu\:P_{l'}(\mu) s(x,\mu)\:.
\]
We use the orthogonality relation
\[
\frac{1}{2} \int_{-1}^1 d\mu\: P_{l'}(\mu) P_l(\mu) = \frac{\delta_{l l'}}{2l+1}
\]
to remove the summation in the collision and scattering terms. \\
However, the streaming term requires more work. We will use the Legendre recursive relationship
\[
\mu P_l(\mu) = \bigl(\frac{l+1}{2l+1}\bigr)P_{l+1}(\mu) + \bigl(\frac{l}{2l+1}\bigr)P_{l-1}(\mu)
\]
to get rid of $\mu$. This all gives, for $l' = 0, 1, \dots$
\[
\bigl(\frac{l'+1}{2l'+1}\bigr)\frac{d}{d x}\phi_{l'+1}(x) + \bigl(\frac{l'}{2l'+1}\bigr)\frac{d}{d x}\phi_{l'-1}(x) + \Sigma_t(x) \phi_{l'} = \Sigma_{sl'}(x)\phi_{l'}(x) + s_{l'}(x)\:.
\]
what do we do when $l'=0$ for $\phi_{-1}$? We typically just set it to $0$, which doesn't matter anyway since we don't technically use that moment, giving
\[
\frac{d}{dx}\phi_1(x) + (\underbrace{\Sigma_t - \Sigma_{s0}}_{\Sigma_a})\phi_0(x) = s_0(x)
\]
This is one equation with two unknowns. Let's try $l=1$:
\[
\frac{2}{3}\frac{d}{dx}\phi_2(x) + \frac{1}{3}\frac{d}{dx}\phi_0(x) + (\Sigma_t - \Sigma_{s1})\phi_1(x) = s_1(x)
\]
Every time we add another equation we get another unknown, so we now have a closure problem.\\
To deal with this, we make the $P_N$ \textit{approximation}. We set either 
\[\phi_{N+1} = 0 \quad\text{ or }\quad\frac{d}{dx}\phi_{N+1}=0
\]
(which gives the same effect) to get $N+1$ coupled equations. 

We can see that diffusion theory is the $P_1$ approximation.\\
If we have an isotropic source, $s_1=0$. Use this in the $l=1$ equation to see
\[
\phi_1(x) = \frac{-1}{3(\Sigma_t - \Sigma_{s1})}\frac{d \phi_0}{dx}
\]
where $\phi_1$ is also known as the net current - so we've got Fick's Law.\\
The $l=0$ equation is then
\[
\frac{-1}{3(\Sigma_t - \Sigma_{s1})}\frac{d^2 \phi_0}{dx^2} + \Sigma_a \phi_0(x) = s_0(x)\:.
\]

In practice, we only ever select $N$ to be odd. This allows us to use $(N+1)/2$ conditions for each of the left and right boundaries. \\
This is much easier than having odd boundary conditions would end up introducing artificial asymmetry. \\
Using $N$ = even doesn't provide more information either. Look at $N=2$ (recall $\phi_3 = 0$):
\begin{align*}
\frac{2}{5}&\frac{d}{dx}\phi_1(x) +(\Sigma_t - \Sigma_{s2})\phi_2(x) = s_2(x)\\
&\\
\phi_2(x) &= \frac{s_2(x) - \frac{2}{5}(\overbrace{s_0(x) - \Sigma_a(x)\phi_0(x)}^{\text{from }l=0})}{\Sigma_t - \Sigma_{s2}}
\end{align*}
$\phi_0$ and $\phi_2$ are linearly dependent, which is not thought to give you better accuracy.\\
When we use $N$ = odd, we don't have this dependence in our last equation. 

We also need to actually set these \textbf{boundary conditions}.
\begin{itemize}
\item \underline{Interface}: $\psi(x,\mu)$ is continuous across every interface. Since
\[
\phi_l = \frac{1}{2}\int_{-1}^1 d\mu\: \mu P_{l}(\mu)\psi(x,\mu)
\]
this condition implies that each $\phi_l(x)$ must also be continuous across that interface. 

\item \underline{Reflective} (extendable to periodic): for some reflecting surface $x_s$:
\begin{align*}
\psi(x_s,\mu) &= \psi(x_s,-\mu) \quad \mu \in [0,1]\\
\sum_{l=0}^{\infty} (2l+1)\phi_l(x_s)P_l(\mu) &= \sum_{l=0}^{\infty} (2l+1)\phi_l(x_s)\underbrace{P_l(-\mu)}_{(-1)^l P_l(\mu)}\\
\sum_{l=0}^{\infty} (2l+1)\phi_l(x_s)P_l(\mu)\bigl[1 - (-1)^l\bigr] &= \sum_{l=1}^{\infty} (2l+1)\phi_l(x_s)\underbrace{P_l(-\mu)}_{(-1)^l P_l(\mu)}\\
&= 0 \quad \text{for odd }l
\end{align*}
Because $P_l(\mu)$ are mutually independent, 
\[\boxed{\phi_l(x_s) =0\text{ for odd }l}\]

\item \underline{Vacuum}: These are trickier in this approximation. If there are no neutrons entering from the right at a boundary $x_0$, the exact condition would be
\[
\psi(x_0, \mu) = 0\:, \quad -1 \leq \mu < 0\:.
\]
However, when we truncate the Legendre expansion we can't meet this exactly (because it implies a discontinuity in the flux at $\mu=0$).\\
Instead, we need other conditions. There are two choices.\\
(1) \textit{Mark BCs}: we set
\[
\psi(x_0, \mu_n) = 0\:, \quad n = 1, 2, \dots, \frac{N+1}{2}
\]
where 
\[
P_{N+1}(\mu_n) = 0\:.
\]
This is the analytical solution of our coupled equations in a source-free, pure absorber from $x_0 \leq x \leq \infty$.

(2) \textit{Marshak BCs}: we simply set the odd angular moments to zero. 
\[
\int_{-1}^0 d\mu\: P_{l}(\mu)\psi(x_0,\mu) = 0\:, \quad n = 1, 3, 5, \dots, N
\]
This condition sets the net number of neutrons entering the solid identically to zero, which we can see from $l=1$:
\[
\int_{-1}^1 d\mu\: \mu \psi(x_0,\mu) = 0\:.
\]
This is the same as saying $J^-(x_0) = 0$ and thus this boundary condition holds even when only using $P_1$. \\
We usually choose the Marshak as it generally introduces less error.  
\end{itemize}

There are some major \textit{deficiencies} of $P_N$. \\
We get poor representation of $\psi$ near vacuum boundaries. We can use Double $P_N$ ($DP_N$) to deal with this.\\
The extension to multi-D is very difficult, and we can use Simplified $P_N$ ($SP_N$) to deal with this.




%For an expanded investigation of this and $DP_N$ see Appendix D of L\&M. \\
%Another method of interest is Simplified $P_N$, $SP_N$. We won't cover this, but I'll post a resource about it on our website for anyone who is curious.


 
%Sorting out discrete ordinates for curvilinear geometries requires some extra work. We are going to skip it to make sure we can through spatial discretization and the basics of transport solvers.


\end{document}
