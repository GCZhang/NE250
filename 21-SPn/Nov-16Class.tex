\documentclass[12pt]{article}
\usepackage[top=1in, bottom=1in, left=1in, right=1in]{geometry}

\usepackage{setspace}
\onehalfspacing

\usepackage{amssymb}
%% The amsthm package provides extended theorem environments
\usepackage{amsthm}
\usepackage{epsfig}
\usepackage{times}
\renewcommand{\ttdefault}{cmtt}
\usepackage{amsmath}
\usepackage{graphicx} % for graphics files
\usepackage{tabu}

% Draw figures yourself
\usepackage{tikz} 

% writing elements
\usepackage{mhchem}

% The float package HAS to load before hyperref
\usepackage{float} % for psuedocode formatting
\usepackage{xspace}

% from Denovo Methods Manual
\usepackage{mathrsfs}
\usepackage[mathcal]{euscript}
\usepackage{color}
\usepackage{array}

\usepackage[pdftex]{hyperref}
\usepackage[parfill]{parskip}

% math syntax
\newcommand{\nth}{n\ensuremath{^{\text{th}}} }
\newcommand{\ve}[1]{\ensuremath{\mathbf{#1}}}
\newcommand{\Macro}{\ensuremath{\Sigma}}
\newcommand{\rvec}{\ensuremath{\vec{r}}}
\newcommand{\vecr}{\ensuremath{\vec{r}}}
\newcommand{\omvec}{\ensuremath{\hat{\Omega}}}
\newcommand{\vOmega}{\ensuremath{\hat{\Omega}}}
\newcommand{\sigs}{\ensuremath{\Sigma_s(\rvec,E'\rightarrow E,\omvec'\rightarrow\omvec)}}
\newcommand{\el}{\ensuremath{\ell}}
\newcommand{\sigso}{\ensuremath{\Sigma_{s,0}}}
\newcommand{\sigsi}{\ensuremath{\Sigma_{s,1}}}
\newcommand{\ep}{\ensuremath{\varepsilon}}
%---------------------------------------------------------------------------
%---------------------------------------------------------------------------
\begin{document}
\begin{center}
{\bf NE 250, F15\\
November 16, 2015 
}
\end{center}


%These notes follow the following reference on SP_N equations:
%McClarren, R.G.
%Theoretical Aspects of the Simplified $P_N$ Equations
%Transport Theory and Statistical Physics, 39:73-109 (2011)

Last time we mentioned that extending $P_N$ to multi-D is very difficult. In
slab geometry the $P_N$ equations can be written as a system of 1-D
diffusion equations; in general geometry this is not possible. 

We can use Simplified $P_N$ ($SP_N$) to deal with this. The $SP_N$ theory is a ``middle ground" between diffusion and transport; it investigates what
would happen if the spherical harmonics ($P_N$) method in general
geometry was as nice as it is in slab geometry.

To formally derive the $SP_N$ equations, we examine the 1-D slab geometry transport equation with an isotropic source:
\[
\mu\frac{\partial\psi}{\partial x} + \Sigma_t\psi = \frac{1}{2}\int_{-1}^1\Sigma_s(\mu_0)\psi(x,\mu')d\mu' + \frac{S}{2},
\]
where $\mu_0 = \omvec\cdot\omvec'$.

The corresponding slab geometry $P_N$ equations are:
\[
\frac{d}{dx}\phi_1(x) + \Sigma_{a0}\phi_0(x) = S(x)
\]
and
\[
\bigl(\frac{l'+1}{2l'+1}\bigr)\frac{d}{d x}\phi_{l'+1}(x) + \bigl(\frac{l'}{2l'+1}\bigr)\frac{d}{d x}\phi_{l'-1}(x) + \Sigma_{al'} \phi_{l'} =  0, \,\,\,\, \text{for $l'>0$}.
\]
[Here, $\Sigma_{al'} = \Sigma_t-\Sigma_{sl'}$].

Now we formally carry out an ad hoc
replacement of terms in the $P_N$. First, for odd values of
$l'$, $\phi_{l'}$ is replaced by a vector:
\[
\phi_{l'}\rightarrow \vec\phi_{l'} = (\phi_{l'}^x,\phi_{l'}^y,\phi_{l'}^z)^t\,\,\,;
\]
then in the even $l'$ equations the derivative in $x$ is replaced by a divergence:
\[
\frac{d}{dx} \rightarrow \nabla \cdot
\]
and in the odd $l'$ equations the $x$ derivative is changed to a gradient:
\[
\frac{d}{dx} \rightarrow \nabla
\]

This allows us to write the first-order form of the $SP_N$ equations as
\[
\begin{split}
\nabla\cdot\vec\phi_1 + \Sigma_{a0}\phi_0 &= S, \\
\bigl(\frac{l'+1}{2l'+1}\bigr)\nabla\phi_{l'+1} + \bigl(\frac{l'}{2l'+1}\bigr)\nabla\phi_{l'-1} + \Sigma_{al'} \vec\phi_{l'} &=  0 \,\,\, \text{for odd $l'$},\\
\bigl(\frac{l'+1}{2l'+1}\bigr)\nabla\cdot\vec\phi_{l'+1} + \bigl(\frac{l'}{2l'+1}\bigr)\nabla\cdot\vec\phi_{l'-1} + \Sigma_{al'} \phi_{l'} &=  0 \,\,\, \text{for even $l'$}.
\end{split}
\]

\begin{itemize}
\item \underline{Marshak BC's} 
\[
\begin{split}
\sum_{l' even}^N\frac{2l'+1}{4\pi}&\phi_{l'}(\vec r)\int_{\hat n\cdot\omvec>0}P_{2l-1}(\hat n\cdot\omvec)P_{l'}(\hat n\cdot\omvec)d^2\omvec \\
& \sum_{l' odd}^N\frac{2l'+1}{4\pi}\hat n\cdot\vec\phi_{l'}(\vec r)\int_{\hat n\cdot\omvec>0}P_{2l-1}(\hat n\cdot\omvec)P_{l'}(\hat n\cdot\omvec)d^2\omvec \\
&\,\,\,\,\,=\int_{\hat n\cdot\omvec>0}P_{2l-1}(\hat n\cdot\omvec)\psi(\vec r,\omvec)d^2\omvec,\\
&\,\,\,\,\,\,\,\,\,\, \text{for $\vec r \in \partial\Gamma$ (in boundary) and $l=1,2,...,(N+1)/2$,} 
\end{split}
\]
where $\hat n$ is the unit inward normal to the boundary $\partial\Gamma$. These boundary conditions are a collection of 1-D Marshak
boundary conditions where the $SP_N$ unknowns are interpreted as
components of a Legendre polynomial expansion.

\item \underline{Interface Conditions} 
The interface conditions for the $SP_N$ equations are found from
the slab geometry interface conditions for odd $N$ to be:
\[
\begin{split}
&\hat n \cdot\phi_1, \\
&l'\phi_{l'-1}+(l'+1)\phi_{l'+1} \,\,\, \text{for $l'$ odd,}\\
&l'(\hat n\cdot\vec\phi_{l'-1})+(l'+1)(\hat n\cdot\vec\phi_{l'+1}) \,\,\, \text{for $l'$ even,}
\end{split}
\]
and
\[
\phi_{N-1}
\]
are continuous at a material interface with outward normal $\hat n$.
\end{itemize}
The simple structure of the $SP_N$ equations can be exploited
to eliminate the vector unknowns. From each odd $l'$ equation we
get 
\[
\vec\phi_{l'}=-\frac{1}{\Sigma_{al'}}\left(\frac{l'+1}{2l'+1}\nabla\phi_{l'+1}+\frac{l'}{2l'+1}\nabla\phi_{l'-1}\right),
\]
assuming $\Sigma_{al'}\neq 0$. Then, using this relation in the even equations,
we get the second-order form of the $SP_N$ equations:
\[
-\nabla\cdot\frac{1}{3\Sigma_{a1}}\nabla\phi_0-\nabla\cdot\frac{2}{3\Sigma_{a1}}\nabla\phi_2+\Sigma_{a0}\phi_0 = S,
\]
and
\[
\begin{split}
-\nabla\cdot&\left(\frac{l'(l'-1)}{(2l'+1)(2l'-1)\Sigma_{al'-1}}\right)\nabla\phi_{l'-2}-\nabla\cdot\left(\frac{(l'+1)(l'+2)}{(2l'+1)(2l'+3)\Sigma_{al'+1}}\right)\nabla\phi_{l'+2}\\
&-\nabla\cdot\left(\frac{l'^2}{(2l'+1)(2l'-1)\Sigma_{al'-1}}+\frac{(l'+1)^2}{(2l'+1)(2l'+3)\Sigma_{al'+1}}\right)\nabla\phi_{l'} +\Sigma_{al'}\phi_{l'} = 0,\\
&\,\,\,\,\, \text{for $l'=2,4,6,...,N-1$.}
\end{split}
\] 
The second-order form is useful because it makes the $SP_N$ equations
look like a set of coupled diffusion equations. 

Let us write the expansions for two and four unknowns to have an idea of what the $SP_N$ equations look like in a concrete sense.
\begin{itemize}
\item \underline{$SP_1$ Equations} 

In first-order form, the $SP_1$ equations are
\[
\begin{split}
&\nabla\cdot\vec\phi_1 + \Sigma_{a0}\phi_0 = S\:,\\
& \frac{1}{3}\nabla\phi_0+(\Sigma_t-\sigsi)\vec\phi_1=0\:,
\end{split}
\]
with boundary conditions given by
\[
\frac{1}{2}\phi_0(\vec r) +\hat n\cdot\vec\phi_1(\vec r) = 2\int_{\hat n\cdot\omvec>0}P_1(\hat n\cdot\omvec)\psi(\vec r,\omvec')d^2\omvec\:, \,\,\,\text{for $\vec r \in \partial\Gamma$}\:.
\]

In second-order form, the $SP_1$ equations are
\[
-\nabla\cdot\frac{1}{3(\Sigma_t-\sigsi)}\nabla\phi_0+\Sigma_{a0}\phi_0 = S,
\]
which is the diffusion approximation to transport in
general geometry. This implies that the $SP_1$ and $P_1$ equations are
the same in general geometry. The boundary condition for the second-order form is
\[
\frac{1}{2}\phi_0(\vec r) -\frac{1}{3(\Sigma_t-\sigsi)}\hat n\cdot\nabla\phi_0(\vec r) = 2\int_{\hat n\cdot\omvec>0}P_1(\hat n\cdot\omvec)\psi(\vec r,\omvec')d^2\omvec\:, \,\,\,\text{for $\vec r \in \partial\Gamma$}\:.
\]

\item \underline{$SP_3$ Equations} 

In first-order form, the $SP_3$ equations are
\[
\begin{split}
&\nabla\cdot\vec\phi_1 + \Sigma_{a0}\phi_0 = S,\\
& \frac{1}{3}\nabla\phi_0+\frac{2}{3}\nabla\phi_2 + \Sigma_{a1}\vec\phi_1=0,\\
& \frac{2}{5}\nabla\cdot\vec\phi_1+\frac{3}{5}\nabla\cdot\vec\phi_3 + \Sigma_{a2}\phi_2=0,\\
& \frac{3}{7}\nabla\phi_2+ \Sigma_{a3}\vec\phi_3=0.
\end{split}
\]
[Keeping in mind that $\Sigma_{al'}=\Sigma_t-\Sigma_{sl'}$].
The boundary conditions can be easily obtained from the general formula given earlier.
%Not worth the time to write them down here

There are two equations in the second-order form of the $SP_3$
equations:
\[
\begin{split}
&-\nabla\cdot\frac{1}{3\Sigma_{a1}}\nabla\phi_0-\nabla\cdot\frac{2}{3\Sigma_{a1}}\nabla\phi_2 + \Sigma_{a0}\phi_0 = S,\\
&-\nabla\cdot\frac{2}{15\Sigma_{a1}}\nabla\phi_0-\nabla\cdot\left(\frac{4}{15\Sigma_{a1}}+\frac{9}{35\Sigma_{a3}}\right)\nabla\phi_2 + \Sigma_{a2}\phi_2 = 0.
\end{split}
\]
The first of these equations is the diffusion equation with a correction
term involving $\phi_2$.

The $SP_3$ equations can be manipulated into a form that resembles
a two group diffusion equation by defining $\hat\phi_0 = \phi_0+2\phi_2$. The first of the two second-order $SP_3$ equations becomes
\[
-\nabla\cdot\frac{1}{3\Sigma_{a1}}\nabla\hat\phi_0 + \Sigma_{a0}\hat\phi_0 = 2\Sigma_{a0}\phi_2+S.
\]
This is a diffusion equation for $\hat\phi_0$ coupled to $\phi_2$ through an interaction term. We can also get such an equation for $\phi_2$:
\[
-\nabla\cdot\frac{9}{35\Sigma_{a3}}\nabla\phi_2+\left(\Sigma_{a2}+\frac{4}{5}\Sigma_{a0}\right)\phi_2 = \frac{2}{5}(\Sigma_{a0}\hat\phi_0-S).
\]

These equations can be solved with a two-group diffusion code
by properly setting the diffusion coefficients and cross-sections or
with a one-group diffusion code utilizing an iteration strategy for
the coupling terms.
%[This iterative strategy is known as the FLIP iteration strategy].
\end{itemize}

When the $SP_N$ equations were first introduced in the 1960's they were not widely accepted as an approximate
transport method because of the lack of a true theoretical
foundation.

This foundation only came in the 1990's with analyses showing the $SP_N$ equations were an asymptotic correction to standard diffusion theory
and asymptotically related to the slab geometry $P_N$ equations.

\begin{itemize}
\item ASYMPTOTIC DERIVATION OF THE $SP_N$ EQUATIONS
\end{itemize}
We will show the most straightforward of these analyses, for isotropic scattering in the one-speed case.

Assume an optically thick system. We scale the transport equation
\[
\omvec\cdot\nabla\psi +\Sigma_t\psi=\frac{\Sigma_s}{4\pi}\phi + \frac{S}{4\pi}.
\]
by
a small, positive, dimensionless parameter $\ep$, such that:
\[
\begin{split}
&\Sigma_t \rightarrow \frac{\Sigma_t}{\ep} \,\,\, \text{(total cross-section is large)}\\
&\Sigma_s \rightarrow \frac{\Sigma_s}{\ep} \,\,\, \text{(same order as total cross-section)}\\
&\Sigma_a \rightarrow \ep^2\Sigma_a \,\,\, \text{(absorption cross-section is small, $O(\ep^2)$)}\\
& S \rightarrow \ep S \,\,\, \text{(source is small)}.
\end{split}
\]

Now the transport equation can be written as:
\[
\left(1+\frac{\ep}{\Sigma_t}\omvec\cdot\nabla\right)\psi = \frac{1-\ep^2\Sigma_a/\Sigma_t}{4\pi}\phi+\frac{\ep^2S}{4\pi\Sigma_t}.
\]

We invert the operator on the left-hand side to get
\[
\psi = \left(1+\frac{\ep}{\Sigma_t}\omvec\cdot\nabla\right)^{-1}\left[\frac{1-\ep^2\Sigma_a/\Sigma_t}{4\pi}\phi+\frac{\ep^2S}{4\pi\Sigma_t}\right].
\]

We expand the inverse operator in a power series:
\[
\begin{split}
\psi = &\left(1-\frac{\ep}{\Sigma_t}\omvec\cdot\nabla+\ep^2\left(\frac{1}{\Sigma_t}\omvec\cdot\nabla\right)^2 -  \ep^3\left(\frac{1}{\Sigma_t}\omvec\cdot\nabla\right)^3 + ... \right.\\
& \left. ... + \ep^6\left(\frac{1}{\Sigma_t}\omvec\cdot\nabla\right)^6 +O(\ep^7)\right)\left[\frac{1-\ep^2\Sigma_a/\Sigma_t}{4\pi}\phi+\frac{\ep^2S}{4\pi\Sigma_t}\right].
\end{split}
\]

Now we integrate over the unit sphere and divide by $4\pi$. Using the identity
\[
\frac{1}{4\pi}\int_{4\pi}\left(\frac{1}{\Sigma_t}\omvec\cdot\nabla\right)^m d^2\omvec = \frac{1+(-1)^m}{2}\frac{1}{m+1}\left(\frac{1}{\Sigma_t}\nabla\right)^m
\]
we get
\[
\begin{split}
\frac{\phi}{4\pi} = \left(1+\frac{\ep^2}{3}\left(\frac{1}{\Sigma_t}\nabla\right)^2 + \frac{\ep^4}{5}\left(\frac{1}{\Sigma_t}\nabla\right)^4 + \frac{\ep^6}{7}\left(\frac{1}{\Sigma_t}\nabla\right)^6+O(\ep^8)\right)\left[\frac{1-\ep^2\Sigma_a/\Sigma_t}{4\pi}\phi+\frac{\ep^2S}{4\pi\Sigma_t}\right].
\end{split}
\]

This gives us
\[
(1-\ep^2\Sigma_a/\Sigma_t)\phi+\frac{\ep^2S}{\Sigma_t}=
\left(1+\frac{\ep^2}{3}\left(\frac{1}{\Sigma_t}\nabla\right)^2 + \frac{\ep^4}{5}\left(\frac{1}{\Sigma_t}\nabla\right)^4 + \frac{\ep^6}{7}\left(\frac{1}{\Sigma_t}\nabla\right)^6+O(\ep^8)\right)^{-1}\phi.
\]
We can \emph{again} expand the right-hand side as a power series to get
\[
(1-\ep^2\Sigma_a/\Sigma_t)\phi+\frac{\ep^2S}{\Sigma_t}=
\left(1-\frac{\ep^2}{3}\left(\frac{1}{\Sigma_t}\nabla\right)^2 - \frac{4\ep^4}{45}\left(\frac{1}{\Sigma_t}\nabla\right)^4 - \frac{44\ep^6}{945}\left(\frac{1}{\Sigma_t}\nabla\right)^6\right)\phi +O(\ep^8).
\]

\begin{itemize}

\item \underline{$SP_1$ Equations}

If we take terms up to $O(\ep^2)$ in the results above, simple algebraic manipulation gives
\[
-\nabla\cdot\frac{1}{3\Sigma_t}\nabla\phi-\Sigma_{a}\phi = S,
\]
which is the second-order form of the $SP_1$ equation.

\item \underline{$SP_3$ Equations}

If we take terms up to $O(\ep^6)$ instead, we get
\[
\frac{\ep^2}{\Sigma_t}(S-\Sigma_a\phi) = -\frac{\ep^2}{3}\left(\frac{1}{\Sigma_t}\nabla\right)^2(\phi+2\phi_2),
\]
where
\[
\phi_2= \frac{2\ep^2}{15}\left(1+\frac{11\ep^2}{21}\left(\frac{1}{\Sigma_t}\nabla\right)^2\right)\left(\frac{1}{\Sigma_t}\nabla\right)^2\phi.
\]

Expanding the first operator in parentheses on the right-hand side we obtain
\[
\phi_2= \frac{2\ep^2}{15}\left(1-\frac{11\ep^2}{21}\left(\frac{1}{\Sigma_t}\nabla\right)^2\right)^{-1}\left(\frac{1}{\Sigma_t}\nabla\right)^2\phi +O(\ep^5).
\]

Dropping the error we rearrange this equation into
\[
-\ep^2\nabla\cdot\frac{1}{\Sigma_t}\nabla\left(\frac{2}{15}\phi+\frac{11}{21}\phi_2\right)+\Sigma_t\phi_2=0,
\]
which is exactly the second-order form of the $SP_3$ equation for $\phi_2$ with isotropic scattering. that we had before. We can also rearrange
\[
\frac{\ep^2}{\Sigma_t}(S-\Sigma_a\phi) = -\frac{\ep^2}{3}\left(\frac{1}{\Sigma_t}\nabla\right)^2(\phi+2\phi_2)
\]
to get the first of the second-order form $SP_3$ equations:
\[
-\nabla\cdot\frac{1}{3\Sigma_t}\nabla(\phi+2\phi_2) + \Sigma_a\phi=S.
\]

This shows that the $SP_3$ equations are a
correction to the diffusion ($SP_1$) equation that is correct through
order $\ep^6$. This means that $SP_3$ equations will have a wider domain
of applicability.

\end{itemize}

There are other ways to derive the $SP_N$ equations, including variational derivations. An additional derivation is on the course website, which came from Tom Evans and Steven Hamilton at ORNL.

There are a variety of conditions under which the $SP_N$ equations and the $P_N$ equations are equivalent.

The $SP_N$ equations can be cast into several alternate but equivalent
forms; of important notice are the $A_N$ equations. It has been shown that the $A_N$ equations are equivalent to the $SP_{2N-1}$ equations.
\end{document}