\documentclass[12pt]{article}
\usepackage[top=1in, bottom=1in, left=1in, right=1in]{geometry}

\usepackage{setspace}
\onehalfspacing

\usepackage{amssymb}
%% The amsthm package provides extended theorem environments
\usepackage{amsthm}
\usepackage{epsfig}
\usepackage{times}
\renewcommand{\ttdefault}{cmtt}
\usepackage{amsmath}
\usepackage{graphicx} % for graphics files

% Draw figures yourself
\usepackage{tikz} 

% writing elements
\usepackage{mhchem}

% The float package HAS to load before hyperref
\usepackage{float} % for psuedocode formatting
\usepackage{xspace}

% from Denovo Methods Manual
\usepackage{mathrsfs}
\usepackage[mathcal]{euscript}
\usepackage{color}
\usepackage{array}

\usepackage[pdftex]{hyperref}
\usepackage[parfill]{parskip}

% math syntax
\newcommand{\nth}{n\ensuremath{^{\text{th}}} }
\newcommand{\ve}[1]{\ensuremath{\mathbf{#1}}}
\newcommand{\Macro}{\ensuremath{\Sigma}}
\newcommand{\rvec}{\ensuremath{\vec{r}}}
\newcommand{\vecr}{\ensuremath{\vec{r}}}
\newcommand{\omvec}{\ensuremath{\hat{\Omega}}}
\newcommand{\vOmega}{\ensuremath{\hat{\Omega}}}
\newcommand{\sigs}{\ensuremath{\Sigma_s(\rvec,E'\rightarrow E,\omvec'\rightarrow\omvec)}}
\newcommand{\el}{\ensuremath{\ell}}
\newcommand{\sigso}{\ensuremath{\Sigma_{s,0}}}
\newcommand{\sigsi}{\ensuremath{\Sigma_{s,1}}}
%---------------------------------------------------------------------------
%---------------------------------------------------------------------------
\begin{document}
\begin{center}
{\bf NE 250, F15\\
September 30, 2015 
}
\end{center}

Announcements:
\begin{itemize}
\item How I grade paragraphs
\item homework due friday; I'll also assign the next one that day
\end{itemize}


\underline{Criticality and Eigenproblems}\\
Loosely from B\&G 1.5.
\begin{itemize}
\item Criticality refers to a self-sustaining \textit{and} controlled chain reaction $\rightarrow$ production = losses
\item subcritical: production $<$ loss $\rightarrow$ chain reaction cannot sustain itself
\item supercritical: production $>$ loss $\rightarrow$ chain reaction increases 
\end{itemize}
%
In the transport model (separation of variables):
\[\psi(\rvec, E, \vOmega, t) = T(t) \Psi(\rvec, E, \vOmega) \:.\]
Plug this in and divide by $\psi$ to get
\[\frac{1}{T}\frac{dT}{dt} = \frac{v}{\Psi} L \Psi(\rvec, E, \vOmega) = \alpha \:.\]
%
Here, $L$ contains streaming, total, scattering, and fission operators. Note that \textit{there is no external source in this equation}, and thus it is homogeneous.

Because the TE is linear, we can use the principle of superposition. 
therefore, the general solution is the sum of \textbf{all} permissible solutions indicated by $\alpha$.

This means time dependence looks like
\[ T(t) = \tau \exp(\alpha t)\]
where $\tau$ is a constant of integration determined by the initial condition.
Then
\[L\Psi = \frac{\alpha}{v}\Psi\]
%
We assume (which is usually true) that the boundary conditions are homogeneous (that is, have no fixed parts).
In this case, we have a linear and homogeneous system.

$\Psi(\rvec, E, \vOmega) = 0$ is a valid solution, but it is the trivial one.

``Eigenvalues" are special values of $\alpha$ for which non-trivial solutions are valid.
for example
\begin{align*}
(L - \frac{\alpha}{v})\Psi &= 0 \:, \quad \in \mathbb{R} \\
\Psi \neq 0 \rightarrow \: &\alpha = v L
\end{align*}
%
The solutions for this problem are eigenfunctions determined to within a multiplicative constant.
If you fine a $\Psi$ that is a solution, then $c \Psi$ is also a solution, where $c$ is a constant:
\[L(c \Psi) = cL \Psi = \frac{\alpha}{v}c \Psi \:.\]

Thus, we want to solve the Eigenproblem: $L\Psi(\rvec, E, \vOmega) = \frac{\alpha}{v}\Psi(\rvec, E, \vOmega)$.\\
A non-trivial solution requires $(L - \frac{\alpha}{v})$ is singular. \\
That is, for $\Psi \neq 0 \rightarrow \: \alpha = v L$, which causes $L - \frac{\alpha}{v}$ to be zero, or non-invertible.


----- aside ----\\
Linear operators: $L(a*x + b*y) = aL(x) + bL(y)$, where $a$ and $b$ and constants and $x$ and $y \: \in \: L$.\\
Examples are the derivative operator and integration. \\
A counter example is a quadratic:
\begin{align*}
L(z) &= z^2 \\
L(a(x) + b(y)) &= (ax + by)^2 = a^2 x^2 + axby + b^2 y^2 \\
& \neq \underbrace{a x^2}_{aL(x)} + \underbrace{b x^2}_{bL(x)}
\end{align*}
-------\\
Solving our equation gives a collection of eigenvalues, $\alpha_j$, and eigenfunctions/modes, $\Psi_j$. 

Our transport operator is linear, so each eval gives an emode, and the solution is a superposition of all emodes. \\
The collection of evals defines the spectrum of $L$.

Superposition of emodes for the general soln of the TE:
\begin{align*}
\psi(\rvec, E, \vOmega, t) &= \sum_{j} \tau_j \exp[\alpha_j t] \Psi_j(\rvec, E, \vOmega) \\
\psi(\rvec, E, \vOmega, 0) &= \sum_{j} \tau_j \Psi_j(\rvec, E, \vOmega) \quad \text{is known.}
\end{align*}

Let us order the eigenvalues such that $Re(\alpha_j) \geq Re(\alpha_{j+1})$. At long times, the solution will be dominated by the $\alpha_0$ term (all components but this one will attenuate). Then, the fundamental emode is correspondingly $\Psi_0$ (which must be real and positive).
\[\lim_{t \to \infty} \psi(\rvec, E, \vOmega, t) = \tau_0 \exp[\alpha_0 t] \Psi_0(\rvec, E, \vOmega)\]
%
\begin{itemize}
\item critical: $\alpha_0 = 0$; the solution is steady state. 
\item subcritical: $\alpha_0 < 0$; the solution diminishes over time.
\item supercritical: $\alpha_0 > 0$; the solution grows over time.
\end{itemize} 
%
%\begin{itemize}
%\item $\alpha_0 = 0$ is the criticality condition; gives a singular operator.
%\item \textit{Given} a composition and geometry, $\alpha_j \rightarrow \psi$ for all time.
%\item \textit{Determine} composition and geometry, $\alpha_0 = 0$ for criticality, $\alpha_j$ are evals.
%\end{itemize}

%------------------------------------------------------------
\vspace*{1 em}
\underline{Existence of SS Soln to TE}

The time independent TE with no external source, $L\Psi_0 = 0$, will be critical if $\alpha_0 = 0$ and will have a unique solution if $\Psi_0$ is unique. \\
More generally, consider the \textit{inhomogenous}, time dependent TE (assumes $L$ and $S$ are time independent):
%
\[\frac{1}{v}\frac{\partial \psi}{\partial t} = L\psi + S = 0\]
%
We're looking for circumstances under which a solution will exist for
\[\frac{1}{v}\frac{\partial \psi}{\partial t} = 0 \quad \text{and} \quad L\psi + S = 0\]

\begin{enumerate}
\item $\alpha_0 < 0$: $\psi$ decays in time and any final solution will be independent of initial conditions since the effect decays with time. If there is a non-zero $\psi$ that balances production and loss (because of the source), a solution exists.
\item $\alpha_0 = 0$: if $S \neq 0$, then ss soln is unbounded; \\
\hspace*{3.5 em}if $S=0$, then ss soln is $\Psi_0$.
\item $\alpha_0 > 0$: a bounded ss soln does \textit{not} exist unless $S=0$ and $\Psi_0 = 0$. That is, no physical solutions can exist in a supercritical system for which $\frac{1}{v}\frac{\partial \psi}{\partial t} = 0$.
\end{enumerate}

\end{document}
