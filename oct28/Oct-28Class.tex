\documentclass[12pt]{article}
\usepackage[top=1in, bottom=1in, left=1in, right=1in]{geometry}

\usepackage{setspace}
\onehalfspacing

\usepackage{amssymb}
%% The amsthm package provides extended theorem environments
\usepackage{amsthm}
\usepackage{epsfig}
\usepackage{times}
\renewcommand{\ttdefault}{cmtt}
\usepackage{amsmath}
\usepackage{graphicx} % for graphics files
\usepackage{tabu}

% Draw figures yourself
\usepackage{tikz} 

% writing elements
\usepackage{mhchem}

% The float package HAS to load before hyperref
\usepackage{float} % for psuedocode formatting
\usepackage{xspace}

% from Denovo Methods Manual
\usepackage{mathrsfs}
\usepackage[mathcal]{euscript}
\usepackage{color}
\usepackage{array}

\usepackage[pdftex]{hyperref}
\usepackage[parfill]{parskip}

% math syntax
\newcommand{\nth}{n\ensuremath{^{\text{th}}} }
\newcommand{\ve}[1]{\ensuremath{\mathbf{#1}}}
\newcommand{\Macro}{\ensuremath{\Sigma}}
\newcommand{\rvec}{\ensuremath{\vec{r}}}
\newcommand{\vecr}{\ensuremath{\vec{r}}}
\newcommand{\omvec}{\ensuremath{\hat{\Omega}}}
\newcommand{\vOmega}{\ensuremath{\hat{\Omega}}}
\newcommand{\sigs}{\ensuremath{\Sigma_s(\rvec,E'\rightarrow E,\omvec'\rightarrow\omvec)}}
\newcommand{\el}{\ensuremath{\ell}}
\newcommand{\sigso}{\ensuremath{\Sigma_{s,0}}}
\newcommand{\sigsi}{\ensuremath{\Sigma_{s,1}}}
%---------------------------------------------------------------------------
%---------------------------------------------------------------------------
\begin{document}
\begin{center}
{\bf NE 250, F15\\
October 28, 2015 
}
\end{center}

Last time we covered a bunch of basics for MC transport: sampling, scoring, and statstics. We ended class with beginning to talk about variance reduction. Recall, we measure improvement by:
\[
FOM =\frac{1}{R^2 t}\:.
\]
\textit{The idea of VR is to track particles that will contribute meaningfully to the desired results and to avoid tracking those that will not while maintaining a fair game.}

\textbf{Categories of VR methods}\\
There are a huge number of VR methods out there. Some are simple, some are complicated. Some are easy to use, others are not. \\
\underline{Misuse of many of these methods can lead to incorrect answers without clear warning that the} \\\underline{answers are incorrect.}\\
We'll go over a list and go into detail of a very few. There are 4 main categories (I'm using the MCNP manual as a reference here):
%
\begin{enumerate}
\item Truncation methods: cut off the parts of phase space you don't think you need
  \begin{itemize}
  \item geometry truncation
  \item energy cutoff
  \end{itemize}
  
\item Population Control methods: directly control the number and weight of particles
  \begin{itemize}
  \item splitting
  \item roulette
  \item executed a variety of ways: geometry-based splitting/roulette, energy-based splitting/roulette, weight windows, weight cutoff
  \end{itemize}

\item Modified Sampling methods: play games with the representation of physics to try to get better particle numbers and weights (alter underlying physical reality without biasing results)
  \begin{itemize}
  \item exponential transform
  \item survival biasing (implicit capture)
  \item forced collisions
  \item source biasing
  \item neutron-induced photon production biasing
  \end{itemize}

\item Partially Deterministic methods [Hazard!]: black magic; circumvent the normal random-walk process by using deterministic-like techniques. These methods can be combined to give results that look good and are completely wrong.
  \begin{itemize}
  \item point detectors
  \item DXTRAN
  \item correlated sampling
  \end{itemize}
\end{enumerate}

\textbf{Survival Biasing}\\
This is a very common technique. In analog MC, what happens when a particle undergoes a collision?
\[
\text{tally }w_i \qquad \text{and } w_{i+1} = 0
\]
If this particle's history is terminated, is it available to contribute to the answer and provide more data?\\
To keep it around, we'll just change the particle's weight in a way that preserves physics rather than terminate it. \\
What is the probability of an absorption during a collision in terms of macroscopic cross sections?
\[
P_{abs} = \frac{\Sigma_a}{\Sigma_t}
\]
We can use this to change particle weight and keep particles around for longer
\[
\text{tally }w_i*\frac{\Sigma_a}{\Sigma_t} \qquad \text{and } w_{i+1} = w_1*(1 - \frac{\Sigma_a}{\Sigma_t})
\]
The reduction in particle weight at each collision compensates, statistically, for the nonphysical scattering. This maintains a fair game and provides more information per history, though at the cost of each collision being worth less.

\textbf{Target Weight Map}\\
We can use this notion of weight to conduct VR. \\
What if we have a weight that is really \textit{low}? : wastes time\\
What if we have a weight that is really \textit{high}? : increases variance (relative error)\\
Ideally, we'd like to keep $w_{\min} \leq w \leq w_{\max}$.

When looking for a specific answer, some regions may be more important to the answer than others. We can make a map expressing the relative importances for a given problem (as we've alluded to). As particles move, they will traverse from regions of one importance to another. We can use how they change importance values to change their weight and the number of particles that we're tracking. We will talk about how you use these maps, and then about some ways to make them.

We can use the idea of importance to create target weights where we'd like particles to exist and associated bounding values. We usually make a map of $w_{nom}$ and set $w_{\min}$ and $w_{\max}$ from there. 

\textbf{Splitting}\\
If particles have too high a weight or are moving into a more important region we can split them into more particles with lower weights. We preserve the total weight to maintain a fare game:\\
\noindent\makebox[\linewidth]{\rule{\textwidth}{0.4pt}}
if $w_i > w_{\max}$:
\begin{itemize}
\item Split Ratio (SR) = $\frac{w_i}{w_{nom}}$
\item get $\xi$, a random number on [0, 1)
\item if $\xi \geq$ (SR - int(SR)):
  \begin{itemize}
  \item create int(SR) new particles
  \end{itemize}
\item else:
  \begin{itemize}
  \item create int(SR) + 1 new particles
  \end{itemize}
\item For all new particles, $w_{i+1} = \frac{w_i}{\# \text{new particles}}$
\end{itemize}
\noindent\makebox[\linewidth]{\rule{\textwidth}{0.4pt}}
%
You're preserving the total weight on average and converting 1 particle with too high a weight to multiple particles with a more useful weight and/or tracking more particles in an important part of the problem. (draw picture)

\textbf{Rouletting}\\
Conversely, if a particle has too low of a weight or is moving into a less important region, we can either increase its weight or kill it. Again, we preserve the total weight.\\
\noindent\makebox[\linewidth]{\rule{\textwidth}{0.4pt}}
if $w_i < w_{\min}$:
\begin{itemize}
\item Roulette Ratio (RR) = $\frac{w_i}{w_{nom}}$
\item get $\xi$, a random number on [0, 1)
\item if $\xi \geq$ RR:
  \begin{itemize}
  \item kill particle; $w_{i+1} = 0$
  \end{itemize}
\item else:
  \begin{itemize}
  \item $w_{i+1} = w_{now}$
  \end{itemize}
\end{itemize}
\noindent\makebox[\linewidth]{\rule{\textwidth}{0.4pt}}
%
On average we're killing enough particles to make up for increasing the weight of some particles. We are taking low weight particles that we don't want to waste our time tracking and converting them to useful particles and/or tracking fewer particles in less important parts of the problem. (draw picture)


%-------------------------------------------------------
-------------------------------------------------------\\
\textbf{General purpose MC codes in nuclear:}
\begin{itemize}
\item \textbf{MCNP}: developed at LANL, distributed via RSICC, \href{http://rsicc.ornl.gov}{http://rsicc.ornl.gov}
\item \textbf{Geant4}: developed by a large collaboration in the HEP community, \href{ http://geant4.web.cern.ch/geant4/}{http://geant4.web.cern.ch/geant4/}
\item \textbf{EGSnrc}: developed at NRC (Canada), \href{http://www.irs.inms.nrc.ca/EGSnrc/EGSnrc.html}{http://www.irs.inms.nrc.ca/EGSnrc/EGSnrc.html}
\item \textbf{SERPENT}: Developed by Dr. Jaakko Leppanen, VTT, Finland, \href{ http://montecarlo.vtt.fi/}{http://montecarlo.vtt.fi/}
\item \textbf{Shift}: developed at ORNL, distributed via RSICC, \href{http://rsicc.ornl.gov}{http://rsicc.ornl.gov}
\end{itemize}






\end{document}
