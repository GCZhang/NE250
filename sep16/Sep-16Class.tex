\documentclass[12pt]{article}
\usepackage[top=1in, bottom=1in, left=1in, right=1in]{geometry}

\usepackage{setspace}
\onehalfspacing

\usepackage{amssymb}
%% The amsthm package provides extended theorem environments
\usepackage{amsthm}
\usepackage{epsfig}
\usepackage{times}
\renewcommand{\ttdefault}{cmtt}
\usepackage{amsmath}
\usepackage{graphicx} % for graphics files

% Draw figures yourself
\usepackage{tikz} 

% writing elements
\usepackage{mhchem}

% The float package HAS to load before hyperref
\usepackage{float} % for psuedocode formatting
\usepackage{xspace}

% from Denovo Methods Manual
\usepackage{mathrsfs}
\usepackage[mathcal]{euscript}
\usepackage{color}
\usepackage{array}

\usepackage[pdftex]{hyperref}
\usepackage[parfill]{parskip}

% math syntax
\newcommand{\nth}{n\ensuremath{^{\text{th}}} }
\newcommand{\ve}[1]{\ensuremath{\mathbf{#1}}}
\newcommand{\Macro}{\ensuremath{\Sigma}}
\newcommand{\rvec}{\ensuremath{\vec{r}}}
\newcommand{\omvec}{\ensuremath{\hat{\Omega}}}
\newcommand{\sigs}{\ensuremath{\Sigma_s(\rvec,E'\rightarrow E,\omvec'\rightarrow\omvec)}}
\newcommand{\el}{\ensuremath{\ell}}
\newcommand{\sigso}{\ensuremath{\Sigma_{s,0}}}
\newcommand{\sigsi}{\ensuremath{\Sigma_{s,1}}}
%---------------------------------------------------------------------------
%---------------------------------------------------------------------------
\begin{document}
\begin{center}
{\bf NE 250, F15 \\
September 16, 2015}
\end{center}

We have now eliminated the dependence of $\phi$ on $\omvec$, so let's take a look at the two equations at
which we arrived last time. In 1D,

\begin{equation*}
\frac{1}{v}\frac{\partial\phi(z,E,t)}{\partial t} = S(z,E,t) + 
\int^{\infty}_0dE'\Sigma_s(E'\rightarrow E)\phi(z,E,t) - 
\Sigma_t(E)\phi(z,E,t) - \frac{\partial}{\partial z}J(z,E,t)
\end{equation*}

\begin{equation*}
\frac{1}{v}\frac{\partial J(z,E,t)}{\partial t} = S_1(z,E,t) + 
\int^{\infty}_0dE'\bar{\mu_0}\Sigma_s(E'\rightarrow E)J(z,E,t) - 
\Sigma_t(E)J(z,E,t) - \frac{1}{3}\frac{\partial}{\partial z}\phi(z,E,t)
\end{equation*}

In 3D,

\begin{equation*}
\frac{1}{v}\frac{\partial\phi(\rvec,E,t)}{\partial t} = S(\rvec,E,t) + 
\int^{\infty}_0dE'\Sigma_s(E'\rightarrow E)\phi(\rvec,E,t) - 
\Sigma_t(E)\phi(\rvec,E,t) - \nabla\cdot\vec{J}(\rvec,E,t)
\end{equation*}

\begin{equation*}
\frac{1}{v}\frac{\partial \vec{J}(\rvec,E,t)}{\partial t} = S_1(\rvec,E,t) + 
\int^{\infty}_0dE'\bar{\mu_0}\Sigma_s(E'\rightarrow E)\vec{J}(\rvec,E,t) - 
\Sigma_t(E)\vec{J}(\rvec,E,t) - \frac{1}{3}\nabla\phi(\rvec,E,t)
\end{equation*}

Assuming that all neutrons have the same energy (``one-speed approximation"),

\begin{equation*}
\frac{1}{v}\frac{\partial\phi(\rvec,t)}{\partial t} = S(\rvec,t) + 
\Sigma_s\phi(\rvec,t) - 
\Sigma_t\phi(\rvec,t) - \nabla\cdot\vec{J}(\rvec,t)
\end{equation*}

\begin{equation*}
\frac{1}{v}\frac{\partial \vec{J}(\rvec,t)}{\partial t} = S_1(\rvec,t) + 
\bar{\mu_0}\Sigma_s\vec{J}(\rvec,t) - 
\Sigma_t\vec{J}(\rvec,t) - \frac{1}{3}\nabla\phi(\rvec,t)
\end{equation*}

Now, assume that the source is isotropic:

\begin{equation*}
S(\rvec,E,\omvec,t) = \frac{S(\rvec,E,t)}{4\pi}
\end{equation*}

\begin{equation*}
S_1(\rvec,E,\omvec,t) = \int_{4\pi}d\omvec S(\rvec,E,\omvec,t)\omvec 
= \frac{S(\rvec,E,t)}{4\pi} \int_{4\pi}d\omvec\omvec = 0
\end{equation*}

Rearranging the current equation gives:

\begin{equation*}
\frac{1}{|\vec{J}(\rvec,t)|}\frac{\partial\vec{J}(\rvec,t)}{\partial t} 
= \frac{\bar{\mu_0}\Sigma_sv\vec{J}(\rvec,t)}{|\vec{J}(\rvec,t)|} - v\Sigma_t 
- \frac{v}{3|\vec{J}(\rvec,t)|}\nabla\phi(\rvec,t)
\end{equation*}

In general, $\frac{1}{|\vec{J}(\rvec,t)|}\frac{\partial\vec{J}(\rvec,t)}{\partial t} \ll v\Sigma_t$, so we
assume/approximate $\frac{1}{|\vec{J}(\rvec,t)|}\frac{\partial\vec{J}(\rvec,t)}{\partial t}\approx 0$. The
collision frequency $v\Sigma_t$ is typically on the order of $10^5$ sec${^-1}$ or larger, so only an
extremely rapid time variation of the current would invalidate this assumption (such rapid changes are
very rarely encountered in reactor dynamics). The current equation then becomes

\begin{equation*}
\vec{J}(\rvec,t) = \frac{-1}{3(\Sigma_t - \bar{\mu_0}\Sigma_s)}\nabla\phi(\rvec,t),
\end{equation*}

which is known as Fick's law. Let us define the diffusion coefficient as

\begin{equation*}
D(\rvec) = \frac{1}{3(\Sigma_t - \bar{\mu_0}\Sigma_s)} = \frac{1}{3\Sigma_{tr}},
\end{equation*}

where $\Sigma_{tr} = \Sigma_t - \bar{\mu_0}\Sigma_s$ is the ``transport" cross section. Also note that

\begin{equation*}
\bar{\mu_0} \approx \frac{2}{3A} > 0,
\end{equation*}

which means that scattering is forward-biased in the lab frame. For large $A$, $\bar{\mu_0}\rightarrow0$,
meaning that scattering is isotropic in the lab frame for large target nuclei. Plugging Fick's law back 
into the flux equation gives

\begin{equation*}
\frac{1}{v}\frac{\partial\phi(\rvec,t)}{\partial t} = S(\rvec,t) - \Sigma_a\phi(\rvec,t) + 
\nabla\cdot[D(\rvec)\nabla\phi(\rvec,t)],
\end{equation*}

which is the one-speed diffusion equation. The diffusion equation is of interest for several reasons. The
$P_1$ approximation is not valid at interface boundaries, in highly absorbing media, or near sources. The
diffusion equation is valid a few mean paths away from a source or a boundary but is also not valid in
highly absorbing media.


Now, assume an isotropic flux $\phi(\rvec,E,\omvec,t)$. The net current, $\vec{J}(\rvec,E,\omvec,t)$, is 
then zero.


Next, consider $J_x(\rvec,E,\omvec,t) > 0, J_y(\rvec,E,\omvec,t) = J_z(\rvec,E,\omvec,t) = 0$.


With the $P_1$ approximation, we have

\begin{equation*}
\phi(\rvec,E,\omvec,t)\approx\frac{1}{4\pi}\phi(\rvec,E,t)+\frac{3}{4\pi}\vec{J}(\rvec,E,t)\cdot\omvec
\end{equation*}

In order for both sides of this equation to be positive, it is required that

\begin{equation*}
-\vec{J}(\rvec,E,t)\cdot\omvec < \frac{1}{3}\phi(\rvec,E,t).
\end{equation*}

For this to be valid, we require that

\begin{equation*}
J_x < \frac{1}{3}\phi(\rvec,E,t).
\end{equation*}

To solve the diffusion equation, we need initial and boundary conditions.


Initial condition: $\phi(\rvec,0) = \phi_0(\rvec) \forall\rvec$


Interface boundary conditions:

% first column
\begin{minipage}[t]{0.5\textwidth}
\underline{Transport} \\
$\phi_1(\rvec_s,E,\omvec,t) = \phi_2(\rvec_s,E,\omvec,t)$ \\
$\forall\rvec_s \in S, S \equiv \partial V, \forall E, \forall \omvec, \forall t$
\end{minipage}
%second column
\begin{minipage}[t]{0.5\textwidth}
\underline{Diffusion} \\
$\phi_1(\rvec_s,t) = \phi_2(\rvec_s,t)$ \\
$\forall\rvec_s \in S, S \equiv \partial V, \forall t$ \\
$\vec{J_1}(\rvec_s,t) = \vec{J_2}(\rvec_s,t)$
\end{minipage}

Vacuum boundary conditions:

% first column
\begin{minipage}[t]{0.5\textwidth}
\underline{Transport} \\
$\phi(\rvec_s,E,\omvec,t) = 0$ \\
$\forall\rvec_s \in S, S \equiv \partial V, \forall E, \forall \omvec: \omvec\cdot\hat{e}_s<0, \forall t$
\end{minipage}
%second column
\begin{minipage}[t]{0.5\textwidth}
\underline{Diffusion} \\
\vspace{-10 mm}
\begin{align*}
\vec{J_-}(\rvec_s) &= \int_{2\pi^-}d\omvec\omvec\cdot\hat{e}_s
\left[\frac{\phi(\rvec,t)}{4\pi} + \frac{3}{4\pi}\omvec\cdot\vec{J}(\rvec,t)\right] \\
&= \frac{1}{4}\phi(\rvec_s,t) + \frac{D(\rvec)}{2}\nabla\phi(\rvec_s,t) \\
&= 0
\end{align*}
\end{minipage}

\begin{equation*}
\vec{J}(\rvec,E,t) = \int_{4\pi}d\omvec\omvec\phi(\rvec,E,\omvec,t) = \text{current at $\rvec$}
\end{equation*}

\begin{equation*}
J = \int_Sds\vec{J}(\rvec,t)\cdot\hat{e}_s = \text{current through a surface $S$}
\end{equation*}

\begin{equation*}
J_{net} = \int_{4\pi}d\omvec\omvec\cdot\hat{e}_s\phi(\rvec,E,\omvec,t) 
= \text{net current at $\rvec$ with respect to $\hat{e}_s$}
\end{equation*}

Scalar currents:

\begin{equation*}
J_{\pm} = \int_{2\pi^{\pm}}d\omvec\omvec\cdot\hat{e}_s\phi(\rvec,E,\omvec,t)
\end{equation*}

\begin{equation*}
2\pi^+ \equiv 0 < \theta < \pi/2, 2\pi^- \equiv \pi/2 < \theta < \pi
\end{equation*}

In 1D,

\begin{equation*}
J_-(z_s,t) = \frac{1}{4}\phi(z_s,t) + \frac{D(z_s)}{2}\frac{d\phi(z_s,t)}{dz}\Bigr|_{z_s} = 0
\end{equation*}

\begin{equation*}
\frac{d\phi(z_s,t)}{dz}\Bigr|_{z_s} = \frac{-\phi(z_s,t)}{2D(z_s)}
\end{equation*}

In order to use the diffusion approximation, we must accept that the above solution at the vacuum is 
wrong, necessitating the introduction of an extrapolation distance:

\begin{equation*}
\tilde{z}_s = z_s + 2D(z_s)
\end{equation*}

This allows us to obtain the correct solution away from the boundary.


And now, back to discussion of the $P_1$ approximation...


In the $P_1$ approximation of the neutron transport equation, the angular flux is linearly anisotropic.
The approximation is valid away from boundaries, away from neutron sources and sinks, and in media that 
are not highly absorbing.

Other assumptions we'll introduce here are the one-speed approximation, an isotropic source term, and the
approximation that $\frac{1}{|\vec{J}(\rvec,t)|}\frac{\partial\vec{J}(\rvec,t)}{\partial t}\ll v\Sigma_t$.
This leads to the one-speed diffusion equation, which was derived above:

\begin{equation*}
\frac{1}{v}\frac{\partial\phi(\rvec,t)}{\partial t} = S(\rvec,t) - \Sigma_a\phi(\rvec,t) + 
\nabla\cdot[D(\rvec)\nabla\phi(\rvec,t)],
\end{equation*}

The boundary conditions necessary to solve this equation are listed above. Now, again consider the 
energy-dependent $P_1$ equations:

\begin{equation*}
\frac{1}{v}\frac{\partial\phi(\rvec,E,t)}{\partial t} = S(\rvec,E,t) + 
\int^{\infty}_0dE'\Sigma_s(E'\rightarrow E)\phi(\rvec,E,t) - 
\Sigma_t(E)\phi(\rvec,E,t) - \nabla\cdot\vec{J}(\rvec,E,t)
\end{equation*}

\begin{equation*}
\frac{1}{v}\frac{\partial \vec{J}(\rvec,E,t)}{\partial t} = S_1(\rvec,E,t) + 
\int^{\infty}_0dE'\bar{\mu_0}\Sigma_s(E'\rightarrow E)\vec{J}(\rvec,E,t) - 
\Sigma_t(E)\vec{J}(\rvec,E,t) - \frac{1}{3}\nabla\phi(\rvec,E,t)
\end{equation*}

From the isotropic source assumption, $S_1(\rvec,E,t) = 0$.


From the $\frac{1}{|\vec{J}(\rvec,t)|}\frac{\partial\vec{J}(\rvec,t)}{\partial t}\ll v\Sigma_t$ 
assumption, $\frac{1}{v}\frac{\partial\vec{J}(\rvec,t)}{\partial t} = 0$.


With the one-speed approximation,

\begin{equation*}
\vec{J}(\rvec,t) \approx -D(\rvec)\nabla\phi(\rvec,t)
\end{equation*}

Similarly,

\begin{equation*}
\vec{J}(\rvec,E,t) \approx -D(\rvec,E)\nabla\phi(\rvec,E,t)
\end{equation*}

Plugging this into the first $P_1$ equation gives

\begin{equation*}
\frac{1}{v}\frac{\partial\phi(\rvec,E,t)}{\partial t} = S(\rvec,E,t) + 
\int^{\infty}_0dE'\Sigma_s(E'\rightarrow E)\phi(\rvec,E,t) - 
\Sigma_t(E)\phi(\rvec,E,t) + \nabla\cdot[D(\rvec,E)\nabla\phi(\rvec,E,t)]
\end{equation*}

Now, let's make the \emph{one-group} (not one-speed) assumption. This means that we will integrate the
entire equation over all energy space $[\int_0^{\infty}dE(\cdot)]$.


``Group constants" are defined as follows:

\begin{equation*}
\Sigma_{t,1}(\rvec)=\frac{\int_0^{\infty}dE\Sigma_t(\rvec,E)\phi(\rvec,E,t)}{\int_0^{\infty}dE\phi(\rvec,E,t)}
= \text{effective cross section}
\end{equation*}

\begin{equation*}
\phi_1(\rvec,t) = \int_0^{\infty}dE\phi(\rvec,E,t) = \text{group flux}
\end{equation*}

Thus, $\int_0^{\infty}dE\Sigma_t(\rvec,E)\phi(\rvec,E,t) = \Sigma_{t,1}(\rvec)\phi_1(\rvec,t)$.

\begin{equation*}
\int_0^{\infty}dE\frac{1}{v}\frac{\partial\phi(\rvec,E,t)}{\partial t} = 
\frac{1}{v_1}\frac{\partial \phi_1(\rvec,t)}{\partial t}, 
\text{ where } \frac{1}{v_1} = \frac{\int_0^{\infty}dE\frac{1}{v}\phi(\rvec,E,t)}{\phi_1(\rvec,t)}
\end{equation*}

\begin{equation*}
\int_0^{\infty}dES(\rvec,E,t) = S_1(\rvec,t)
\end{equation*}

\begin{equation*}
\int_0^{\infty}dE\int_0^{\infty}dE'\Sigma_s(E'\rightarrow E)\phi(\rvec,E',t) =
\int_0^{\infty}dE'\Sigma_s(E')\phi(\rvec,E',t) = \sigsi\phi_1
\end{equation*}

\begin{equation*}
\int_0^{\infty}dE\nabla\cdot[D(\rvec,E)\nabla\phi(\rvec,E,t)]=\nabla\cdot[D_1(\rvec)\nabla\phi_1(\rvec,t)]
\end{equation*}

Note that although the subscripts used here for the one-group approximation are the same as those used in
the $P_1$ approximation derivation, the quantities are not the same. Combining all of the above terms, we
have

\begin{equation*}
\frac{1}{v_1}\frac{\partial \phi_1(\rvec,t)}{\partial t} = S_1(\rvec,t) - 
\Sigma_{a,1}(\rvec)\phi_1(\rvec,t) + \nabla\cdot[D_1(\rvec)\nabla\phi_1(\rvec,t)]
\end{equation*}

This is the \emph{one-group} diffusion equation.

\end{document}
