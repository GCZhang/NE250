\documentclass[12pt]{article}
\usepackage[top=1in, bottom=1in, left=1in, right=1in]{geometry}

\usepackage{setspace}
\onehalfspacing

\usepackage{amssymb}
%% The amsthm package provides extended theorem environments
\usepackage{amsthm}
\usepackage{epsfig}
\usepackage{times}
\renewcommand{\ttdefault}{cmtt}
\usepackage{amsmath}
\usepackage{graphicx} % for graphics files

% Draw figures yourself
\usepackage{tikz} 

% writing elements
\usepackage{mhchem}

% The float package HAS to load before hyperref
\usepackage{float} % for psuedocode formatting
\usepackage{xspace}

% from Denovo Methods Manual
\usepackage{mathrsfs}
\usepackage[mathcal]{euscript}
\usepackage{color}
\usepackage{array}

\usepackage[pdftex]{hyperref}
\usepackage[parfill]{parskip}

% math syntax
\newcommand{\nth}{n\ensuremath{^{\text{th}}} }
\newcommand{\ve}[1]{\ensuremath{\mathbf{#1}}}
\newcommand{\Macro}{\ensuremath{\Sigma}}
\newcommand{\rvec}{\ensuremath{\vec{r}}}
\newcommand{\omvec}{\ensuremath{\hat{\Omega}}}
\newcommand{\sigs}{\ensuremath{\Sigma_s(\rvec,E'\rightarrow E,\omvec'\rightarrow\omvec)}}
\newcommand{\el}{\ensuremath{\ell}}
\newcommand{\sigso}{\ensuremath{\Sigma_{s,0}}}
\newcommand{\sigsi}{\ensuremath{\Sigma_{s,1}}}
%---------------------------------------------------------------------------
%---------------------------------------------------------------------------
\begin{document}
\begin{center}
{\bf NE 250, F15 \\
September 14, 2015}
\end{center}

In solving the neutron transport equation, we're interested in solving for $\phi(\rvec, E, \omvec, t)$.


One quantity in the neutron transport equation to be considered is the macroscopic cross section. We'll
start off by noting that the microscopic cross section is a function of only energy [$\sigma(E))$] while
the macroscopic cross section depends on both space and energy [$\Sigma(\rvec,E) = N(\rvec)\sigma(E)$].
This is because the macroscopic cross section is for an entire material and must take the number density
of that material into account, which introduces a spatial dependence.


Absorption reactions (radiative capture, fission, etc.) are assumed to be isotropic (that is, independent
of $\omvec$), but this assumption doesn't hold for scattering: $\sigs$.


Additionally, leakage strongly depends on $\omvec$. Since we're solving for $\phi(\rvec, E, \omvec, t)$
and that solution is made difficult by other angularly-dependent variables, let's try to eliminate the
dependence of $\phi$ on $\omvec$. One approach to this simplification is the $S_N$ (discrete ordinates)
method, which changes $\omvec$ from continuous to discrete ($\omvec \rightarrow \omvec_n; n = 1,\dots,N)$.
This also discretizes $\phi$: $\phi_n = \phi(\omvec_n)$.


This discretization changes the integral of $\phi$ over $\omvec$ to a sum:

\begin{equation*}
\int d\omvec\phi(\omvec) \rightarrow \sum^{N}_{n=1}w_n\phi_n,
\end{equation*}

where $w_n$ is the $n^{th}$ weighting factor. The neutron transport equation then becomes

\begin{equation*}
\frac{1}{v}\frac{\partial\phi_n}{\partial t} = 
S_n + \int^{\infty}_{0}dE'\sum_{n'=1}^{N}\sigs w_n\phi_n-\Sigma_t\phi_n
-\omvec_n\cdot\nabla\phi_n.
\end{equation*}

A second type of discretization is the $P_N$ method, which does a series expansion of $\phi$:

\begin{equation*}
\phi(\rvec,E,\omvec,t) =\sum_{\el=0}^{\infty}\sum_{m=-\ell}^{\ell}Y_{\el m}(\omvec)\phi(\el,m)(\rvec,E,t),
\end{equation*}

where $Y_{\el m}(\omvec)$ is a spherical harmonic. Notes on spherical harmonics can be found in Appendix D
of ``Computational Methods of Neutron Transport" by Lewis and Miller.


From here, let's assume 1D plane symmetry for the sake of simplicity. Using the $P_N$ method, we have

\begin{equation*}
\phi(z,E,\mu,t) = \sum_{\el=0}^{\infty}\frac{2\el+1}{2}P_{\el}(\mu)\phi_{\el}(z,E,t),
\end{equation*}

where $P_{\ell}(\mu)$ is a Legendre polynomial. Important notes on Legendre polynomials:
\begin{gather*}
P_0(\mu) = 1 \\
P_1(\mu) = \mu \\
\int_{-1}^{1}P_n(\mu)P_m(\mu)d\mu = 
\frac{2}{2n+1}\delta_{n,m}=\frac{2}{2n+1}\text{if n = m, 0 if n $\neq$ m.}
\end{gather*}

Using $N=1$ with Legendre polynomials (the ``$P_1$ method") gives

\begin{equation*}
\phi(z,E,\mu,t) = \frac{1}{2}P_0(\mu)\phi_0(z,E,t) + \frac{3}{2}P_1(\mu)\phi_1(z,E,t).
\end{equation*}

In general, (where did this come from???)

\begin{equation*}
\phi_{\el}(z,E,t) = \int^{1}_{-1}d\mu\phi(z,E,\mu,t)P_{\el}(\mu) 
= \sum_{\el'=0}^{\infty}\frac{2\el'+1}{2}P_{\ell'}(\mu)\phi_{\el'}P_{\ell}(\mu).
\end{equation*}

Plugging the new approximate definition of $\phi$ into the transport equation gives

\begin{multline*}
\frac{1}{v}\frac{\partial\phi_n}{\partial t}\left[\frac{1}{2}P_0(\mu)\phi_0(z,E,t) + \frac{3}{2}P_1(\mu)\phi_1(z,E,t)\right] = 
\frac{1}{2}P_0(\mu)S_0 + \frac{3}{2}P_1(\mu)S_1 + \\ 
\int^{\infty}_{0}dE'\int^{1}_{-1}d\mu'\sigs\left[\frac{1}{2}P_0(\mu')\phi_0(z,E,t) + \frac{3}{2}P_1(\mu')\phi_1(z,E,t)\right] \\
- \Sigma_t\left[\frac{1}{2}P_0(\mu)\phi_0(z,E,t) + \frac{3}{2}P_1(\mu)\phi_1(z,E,t)\right]
- \mu\frac{\partial}{\partial z}\left[\frac{1}{2}P_0(\mu)\phi_0(z,E,t) + \frac{3}{2}P_1(\mu)\phi_1(z,E,t)\right]
\end{multline*}

Now, let's take the zeroth moment of this equation (that is, multiply it by $P_0(\mu)$) and integrate the
equation over $\int_{-1}^{1}d\mu$. Let $\mu_s = cos(\theta'-\theta) = \omvec'\cdot\omvec$ such that
$\Sigma_s(\mu'\rightarrow\mu) = \Sigma_s(\mu_s)$. Then,

\begin{equation*}
\Sigma_s(\mu_s) = \frac{1}{2}P_0(\mu_s)\sigso + \frac{3}{2}P_1(\mu_s)\sigsi
\end{equation*}

\begin{multline*}
\int_{-1}^{1}d\mu\int_{-1}^{1}d\mu'\left[\frac{1}{2}P_0(\mu_s)\sigso+\frac{3}{2}P_1(\mu_s)\sigsi\right]
\left[\frac{1}{2}P_0(\mu')\phi_0(z,E,t)+\frac{3}{2}P_1(\mu')\phi_1(z,E,t)\right]P_0(\mu) = \\
\int_{-1}^{1}d\mu\int_{-1}^{1}d\mu'\frac{1}{2}P_0(\mu_s)\sigso\frac{1}{2}P_0(\mu')\phi_0(z,E,t)P_0(\mu) =
\sigso\phi_0(z,E,t)
\end{multline*}

\begin{equation*}
\int_{-1}^{1}d\mu\mu\left[\frac{1}{2}P_0(\mu)\phi_0(z,E,t)+\frac{3}{2}P_1(\mu)\phi_1(z,E,t)\right]P_0(\mu)
= \phi_1
\end{equation*}

This gives us

\begin{equation*}
\frac{1}{v}\frac{\partial\phi_0}{\partial t} = S_0 + \sigso\phi_0 - \Sigma_t\phi_0 - 
\frac{\partial\phi_1}{\partial z}
\end{equation*}

Now, let's take the first moment of the transport equation with the approximate $\phi$ definition and 
integrate it over $\int_{-1}^{1}d\mu$:

\begin{multline*}
\int_{-1}^{1}d\mu\int_{-1}^{1}d\mu'\left[\frac{1}{2}P_0(\mu_s)\sigso+\frac{3}{2}P_1(\mu_s)\sigsi\right]
\left[\frac{1}{2}P_0(\mu')\phi_0(z,E,t)+\frac{3}{2}P_1(\mu')\phi_1(z,E,t)\right]P_1(\mu) = \\
\int_{-1}^{1}d\mu\int_{-1}^{1}d\mu'\frac{3}{2}P_1(\mu_s)\sigsi\frac{3}{2}P_1(\mu')\phi_1(z,E,t)P_1(\mu) =
\sigsi\phi_1(z,E,t)
\end{multline*}

\begin{equation*}
\int_{-1}^{1}d\mu\mu\left[\frac{1}{2}P_0(\mu)\phi_0(z,E,t)+\frac{3}{2}P_1(\mu)\phi_1(z,E,t)\right]P_1(\mu)
= \frac{1}{3}\phi_0
\end{equation*}

This gives us

\begin{equation*}
\frac{1}{v}\frac{\partial\phi_1}{\partial t} = S_1 + \sigsi\phi_1 - \Sigma_t\phi_1 - 
\frac{1}{3}\frac{\partial\phi_0}{\partial z}
\end{equation*}

Scalar flux:

\begin{equation*}
\phi(z,E,t) = \phi_0(z,E,t) = \int_{-1}^{1}d\mu\phi(z,E,\mu,t)P_0(\mu) = \int_{-1}^{1}d\mu\phi(z,E,\mu,t)
\end{equation*}

Scalar current:

\begin{equation*}
J(z,E,t) = \phi_1(z,E,t) = \int_{-1}^{1}d\mu\phi(z,E,\mu,t)P_1(\mu) = \int_{-1}^{1}d\mu\mu\phi(z,E,\mu,t)
\end{equation*}

Neutron source(s) summed over every direction:

\begin{equation*}
S(z,E,t) = S_0(z,E,t) = \int_{-1}^{1}d\mu S(z,E,\mu,t)
\end{equation*}

\begin{equation*}
S_1(z,E,t) = \int_{-1}^1d\mu S(z,E,\mu,t)P_1(\mu)
\end{equation*}

Scattering cross sections:

\begin{gather*}
\sigso = \Sigma_s(E') \\
\sigsi = \int_{-1}^1d\mu_s\Sigma_s(\mu_s)P_1(\mu_s) \\
\frac{d\mu_s\Sigma_s(\mu_s)}{\Sigma_s} = \text{probability that neutron will scatter in $d\mu_s$ about $\mu_s$} \\
\sigsi = \int_{-1}^1d\mu_s\frac{\Sigma_s\mu_s}{\Sigma_s}\mu_s\Sigma_s = \bar{\mu_0}\Sigma_s, 
\text{where $\bar{\mu_0} =$ average scattering angle}
\end{gather*}

<fix these last notes about the scattering term>

\end{document}
